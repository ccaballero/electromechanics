\chapter{Transformada inversa de \emph{Laplace}}

Si $\mathcal{L}\{f(t)\}=F(s)$:
\begin{equation}
    \mathcal{L}^{-1}\{F(s)\}=f(t)
\end{equation}

Definida para $t>0$.

La transformada inversa de \emph{Laplace} regresa del dominio de ``s'' al
dominio de ``t''.

\textbf{Ejemplo}:
\begin{equation*}
    \mathcal{L}^{-1}\biggl\{\frac{1}{s^n}\biggl\}
\end{equation*}
\begin{equation*}
    \mathcal{L}\{t^n\}=\frac{\Gamma(n+1)}{s^{n+1}}
\end{equation*}
\begin{equation*}
    t^n=\mathcal{L}^{-1}\biggl\{\frac{\Gamma(n+1)}{s^{n+1}}\biggl\}
\end{equation*}
\begin{equation*}
    \frac{t^n}{\Gamma(n+1)}=\mathcal{L}^{-1}\biggl\{\frac{1}{s^{n+1}}\biggl\}
\end{equation*}
\begin{equation}
    \mathcal{L}^{-1}\biggl\{\frac{1}{s^n}\biggl\}=\frac{t^{n-1}}{\Gamma(n)}
\end{equation}

En particular $n\in\mathbb{N}$:
\begin{equation}
    \mathcal{L}^{-1}\biggl\{\frac{1}{s^n}\biggl\}=\frac{t^{n-1}}{(n-1)!}
\end{equation}

\section{Tabla de transformadas de \emph{Laplace} inversas}

\begin{equation*}
\def\arraystretch{1.4}
\begin{array}{@{}cll@{}}
\toprule
 & F(s) & f(t)=\mathcal{L}^{-1}\{F(s)\};t>0\\
\cmidrule(l){1-3}
 1 & \dfrac{k}{s}
   & k\\
\cmidrule(l){1-3}
 2 & \dfrac{1}{s^n}
   & \dfrac{t^{n-1}}{\Gamma(n)};\quad\dfrac{t^{n-1}}{(n-1)!}\,n\in\mathbb{N}\\
\cmidrule(l){1-3}
 3 & \dfrac{1}{s-a}
   & e^{at}\\
\cmidrule(l){1-3}
 4 & \dfrac{1}{s^2+a^2}
   & \dfrac{1}{a}\sen(at)\\
\cmidrule(l){1-3}
 5 & \dfrac{s}{s^2+a^2}
   & \cos(at)\\
\cmidrule(l){1-3}
 6 & \dfrac{1}{s^2-a^2}
   & \dfrac{1}{a}\senh(at)\\
\cmidrule(l){1-3}
 7 & \dfrac{s}{s^2-a^2}
   & \cosh(at)\\
\cmidrule(l){1-3}
 8 & \arctan\left(\dfrac{a}{s}\right)
   & \dfrac{1}{t}\sen(at)\\
\cmidrule(l){1-3}
 9 & k
   & k\delta(t)\\
\cmidrule(l){1-3}
10 & e^{-as}
   & \delta(t-a)\\
\bottomrule
\end{array}
\end{equation*}

\section{Propiedades de la transformada inversa de \emph{Laplace}}

\subsection{Linealidad}
\begin{equation}
    \mathcal{L}^{-1}\{a_1\,F_1(s)+a_2\,F_2(s)\}
    =a_1\,f_1(t)+a_2\,f_2(t)
\end{equation}

\subsection{Desplazamiento en $t$}
Si $\mathcal{L}^{-1}\{F(s)\}=f(t)$:
\begin{equation}
    \mathcal{L}^{-1}\{F(s)\,e^{-as}\}=f(t-a)u(t-a)
\end{equation}

\subsection{Desplazamiento en $s$}
Si $\mathcal{L}^{-1}\{F(s)\}=f(t)$:
\begin{equation}
    \mathcal{L}^{-1}\{F(s-a)\}=f(t)\,e^{at}
\end{equation}

\subsection{División por $s$}
Si $\mathcal{L}^{-1}\{F(s)\}=f(t)$:
\begin{equation}
    \mathcal{L}^{-1}\biggl\{\frac{F(s)}{s}\biggl\}=\int_0^t\,f(t)\,dt
\end{equation}

En general:
\begin{equation}
    \mathcal{L}^{-1}\biggl\{\frac{F(s)}{s^n}\biggl\}
        =\int_0^t\int_0^t\dots\int_0^t\,f(t)\,dt\dots\,dt\,dt
\end{equation}

\subsection{Transformada inversa de la derivada}
Si $\mathcal{L}^{-1}\{F(s)\}=f(t)$:
\begin{equation}
    \mathcal{L}^{-1}\{F'(s)\}=-t\,f(t)
\end{equation}

En general:
\begin{equation}
    \mathcal{L}^{-1}\{F^{(n)}(s)\}={(-1)}^n\,t^n\,f(t)
\end{equation}

\section{Descomposición en fracciones parciales}
En la mayoría de los casos se tienen funciones racionales de la siguiente forma:

\begin{equation*}
    F(s)=\frac{P(s)}{Q(s)}
\end{equation*}

Donde el grado del numerador debe ser menor al grado del denominador. Para
hallar la transformada inversa se debe descomponer en fracciones parciales de
acuerdo a los factores del denominador. Se tienen los siguientes casos:

\subsection{Factores lineales no repetidos}
\begin{equation*}
    \frac{P(s)}{(s-a_1)(s-a_2)\dots(s-a_n)}
        =\frac{A_1}{s-a_1}+\frac{A_2}{s-a_2}+\cdots+\frac{A_n}{s-a_n}
\end{equation*}

\subsection{Factores lineales y repetidos}
\begin{equation*}
    \frac{P(s)}{{(s-a)}^m{(s-b)}^n}
        =\frac{A_1}{s-a}+\frac{A_2}{{(s-a)}^2}+\cdots+\frac{A_m}{{(s-a)}^m}
        +\frac{B_1}{(s-b)}+\frac{B_2}{{(s-a)}^2}+\cdots+\frac{B_n}{{(s-b)}^n}
\end{equation*}

\subsection{Factores cuadráticos (con raíces imaginarias o complejas)}
\begin{equation*}
    \frac{P(s)}{(s^2+{a_1}s+b_1)(s^2+{a_2}s+b_2)}
        =\frac{{A_1}s+B_1}{s^2+{a_1}s+b_1}
        +\frac{{A_2}s+B_2}{s^2+{a_2}s+b_2}
\end{equation*}

\section{Convolución}
\begin{equation*}
    f_1(t)*f_2(t)=\int_0^t\,f_1(\tau)f_2(t-\tau)\,d\tau
\end{equation*}
\begin{equation*}
\begin{split}
    \mathcal{L}\{f_1(t)*f_2(t)\}
        &=\int_0^{\infty}\left[\int_0^t\,f_1(\tau)f_2(t-\tau)\,d\tau\right]e^{-st}\,dt\\
        &=\int_{t=0}^{t=\infty}\int_{\tau=0}^{\tau=t}\,f_1(\tau)f_2(t-\tau)e^{-st}\,d\tau\,dt\\
        &=\lim_{M\rightarrow\infty}\int_{\tau=0}^{\tau=M}\int_{t=\tau}^{t=M}\,f_1(\tau)f_2(t-\tau)e^{-st}\,dt\,d\tau\\
        &=\lim_{M\rightarrow\infty}\int_0^M\,f_1(\tau)\,d\tau\int_{\tau}^{M}\,f_2(t-\tau)e^{-st}\,dt
\end{split}
\end{equation*}
\begin{equation*}
    u=t-\tau\rightarrow\,t=u+\tau
\end{equation*}
\begin{equation*}
    du=dt
\end{equation*}
\begin{equation*}
\begin{split}
    \mathcal{L}\{f_1(t)*f_2(t)\}
        &=\lim_{M\rightarrow\infty}\int_0^M\,f_1(\tau)\,d\tau\int_{0}^{M-\tau}\,f_2(u)e^{-s(u+\tau)}\,du\\
        &=\int_0^{\infty}\,f_1(\tau)e^{-st}\,d\tau\int_0^{\infty}\,f_2(u)e^{-su}\,du\\
        &=F_1(s)\,F_2(s)
\end{split}
\end{equation*}
\begin{equation}
    \mathcal{L}\{f_1(t)*f_2(t)\}=F_1(s)\,F_2(s)
\end{equation}


\section{Transformada inversa por convolución}
\begin{equation}
    \mathcal{L}^{-1}\{F_1(s)\,F_2(s)\}=f_1(t)*f_2(t)
\end{equation}

Donde:
\begin{equation*}
    f_1(t)=\mathcal{L}^{-1}\{F_1(s)\}
\end{equation*}
\begin{equation*}
    f_2(t)=\mathcal{L}^{-1}\{F_2(s)\}
\end{equation*}

