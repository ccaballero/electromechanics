\chapter{Sistemas conservativos}

\section{Ejemplos}

\subsection{Ejemplo 1}
Determine las ecuaciones de movimiento y las frecuencias naturales del sistema
masa-resorte de dos grados de libertad utilizando la función de \emph{Lagrange}
$L$.

\subsection{Ejemplo 2}
Utilizar la función de \emph{Lagrange} $L$ para deducir las ecuaciones de
movimiento del péndulo acoplado.

\subsection{Ejemplo 3}
Determinar las ecuaciones de movimiento de un péndulo suspendido en una banda de
caucho utilizando la función de \emph{Lagrange}.

\subsection{Ejemplo 4}
5.6. La barra uniforme $AB$, de masa $M$ y longitud $l$, esta sostenida por la
articulación lisa en $A$. El extremo $B$ se encuentra unido al resorte $BC$, el
cual esta aun estirado cuando $\theta=0$. Despreciando la masa del resorte,
demostrar que para este sistema:

\begin{equation*}
    V=-\frac{1}{2}Mgl\cos(\theta)+
      \frac{1}{2}k{[{(s^2+l^2-2sl\cos(\theta))}^{\frac{1}{2}}-l_0]}^2
\end{equation*}

Suponiendo que $\theta$ es una ángulo pequeño, aproximar el valor de la energía
potencial $V$, por medio del desarrollo de \emph{Taylor} y determinar el periodo
de oscilación de la varilla.

\subsection{Ejemplo 5}
5.13. Tres piñones $G_1$, $G_2$ y $G_3$ se encuentran acoplados por medio de
resortes de torsión a los discos $D_1$, $D_2$ y $D_3$. Demostrar que la función
de \emph{Lagrange} para el sistema es:

\begin{equation*}
\begin{split}
    L=\frac{I_1}{2}\dot{\theta}_1^2+
      \frac{d_2^2}{2}\left(
        \frac{I_3}{d_1^2}+\frac{I_4}{d_2^2}+\frac{I_5}{d_3^2}
      \right)\dot{\theta}_4^2+
      \frac{I_6}{2}\dot{\theta}_6^2+
      \frac{I_7}{2}\dot{\theta}_7^2-
      \frac{k_1}{2}{(\theta_4-\theta_1)}^2\\
      -\frac{k_2 d_2^2}{2}{\left(
        \frac{\theta_4}{d_1}-\frac{\theta_6}{d_2}
      \right)}^2-
      \frac{k_3 d_2^2}{2}{\left(
        \frac{\theta_4}{d_3}-\frac{\theta_7}{d_2}
      \right)}^2
\end{split}
\end{equation*}

\subsection{Ejemplo 6}
En la figura esta simbolizado esquemáticamente un típico caso de un grupo de
maquinas conformado por un motor conectado con acople elástico con la maquina
que realiza trabajo. Si se ejecuta la reducción de las cantidades másicas, por
ejemplo sobre el eje del motor, se obtienen los citados sistemas de dos masas,
en donde $I_1$ e $I_2$ son los momentos inerciales reducidos del motor y de la
maquina, y $c$ es la constante de rigidez del acople elástico. Establecer la
frecuencia natural y las principales formas de las oscilaciones propias y
expresar el desarrollo de las funciones $\psi_1(t)$ y $\psi_2(t)$. Utilizar la
función de \emph{Lagrange} $L$.

\subsection{Ejemplo 7}
El chasis de un automóvil se puede simbolizar esquemáticamente. Si $I_s$
significa el momento inercial del chasis con respecto al eje que pasa por el
centro de gravedad perpendicular al esquema, y $c_1=c_2=c$, el coeficiente de
rigidez de los resortes del chasis, determinar la frecuencia de oscilación
natural del sistema.

