\documentclass[letter,11pt]{article}

\usepackage[spanish,es-nodecimaldot]{babel}
\usepackage[utf8]{inputenc}

\usepackage{helvet}
\renewcommand{\familydefault}{\sfdefault}
\usepackage[T1]{fontenc}
\usepackage{textcomp}

\usepackage{framed}
\usepackage[svgnames]{xcolor}
\colorlet{shadecolor}{Gainsboro!50}

\usepackage[shortlabels]{enumitem}
\usepackage{graphicx}
\usepackage{pstricks}

\usepackage{anysize}
\marginsize{3cm}{2cm}{2cm}{3cm}

\usepackage{siunitx}
\usepackage{amsmath}
\usepackage{array}
\usepackage{alltt}

\usepackage{fancyhdr}
\usepackage{lastpage}
\pagestyle{fancy}
\fancyhf{}
\fancyhead[LE,RO]{Física Básica III}
\fancyfoot[CO,CE]{\thepage\ de \pageref{LastPage}}

\special{papersize=215.9mm,279.4mm}

\usepackage[
    pdfauthor={Carlos Eduardo Caballero Burgoa},%
    pdftitle={Física Básica III},%
    pdfsubject={Tarea},%
    colorlinks,%
    citecolor=black,%
    filecolor=black,%
    linkcolor=black,%
    urlcolor=black,
    breaklinks]{hyperref}
\usepackage{breakurl}

\newcommand{\blankpage}{
\newpage
\thispagestyle{empty}
\mbox{}
\newpage
}

\renewcommand{\arraystretch}{1.2}
\newcommand{\pvec}[1]{\vec{#1}\mkern2mu\vphantom{#1}}

\begin{document}

\begin{center}
    {\Large \bf{\underline{FUERZA Y CAMPO ELÉCTRICO}}}
\end{center}
\vspace{0.5cm}

\begin{equation*}
    \vec{F} = \frac{1}{4 \pi \epsilon_0} \frac{Q Q_1}{|\vec{r}|^2} \hat{r}
\end{equation*}
\begin{equation*}
    \vec{F} = \frac{1}{4 \pi \epsilon_0} \frac{Q Q_1}{|\vec{r}-\vec{r}_1|^2} \hat{r}
\end{equation*}
\begin{equation*}
    \vec{F} = \frac{1}{4 \pi \epsilon_0} \frac{Q Q_1}{|\vec{r}-\vec{r}_1|^2} \left(\frac{\vec{r}-\vec{r}_1}{|\vec{r}-\vec{r}_1|}\right)
\end{equation*}
\begin{equation*}
    \vec{F} = \frac{1}{4 \pi \epsilon_0} Q Q_1 \left(\frac{\vec{r}-\vec{r}_1}{|\vec{r}-\vec{r}_1|^3}\right)
\end{equation*}
\begin{equation*}
    \vec{E} = \frac{1}{q} \vec{F}
\end{equation*}
\begin{equation*}
    \vec{F} = \frac{1}{4 \pi \epsilon_0} \sum_{i=1}^n Q Q_i \left(\frac{\vec{r}-\vec{r}_i}{|\vec{r}-\vec{r}_i|^3}\right)
\end{equation*}
\begin{equation*}
    \vec{E} = \frac{1}{4 \pi \epsilon_0} \sum_{i=1}^n Q_i \left(\frac{\vec{r}-\vec{r}_i}{|\vec{r}-\vec{r}_i|^3}\right)
\end{equation*}

\begin{equation*}
    \vec{F} = \frac{1}{4 \pi \epsilon_0} \int_Q Q dq \left(\frac{\vec{r}-\vec{r}_i}{|\vec{r}-\vec{r}_i|^3}\right)
\end{equation*}
\begin{equation*}
    \vec{E} = \frac{1}{4 \pi \epsilon_0} \int_Q dq \left(\frac{\vec{r}-\vec{r}_i}{|\vec{r}-\vec{r}_i|^3}\right)
\end{equation*}

\begin{equation*}
    \lambda = \frac{q}{L} = \frac{dq}{dx}
\end{equation*}
\begin{equation*}
    \sigma = \frac{q}{A} = \frac{dq}{dx\,dy}
\end{equation*}
\begin{equation*}
    \rho = \frac{q}{V} = \frac{dq}{dx\,dy\,dz}
\end{equation*}

\end{document}

