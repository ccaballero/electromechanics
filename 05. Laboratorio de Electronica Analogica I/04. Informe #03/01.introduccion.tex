\section{Introducción}
Dos tipos básicos de transistores son el transistor de unión bipolar (BJT,
\emph{bipolar junction transistor}) y el transistor de efecto de campo (FET,
\emph{field-effect transistor}, estos transistores pueden usarse como
amplificadores de una señal eléctrica.

Un amplificador es un dispositivo que tiene la capacidad de aumentar la fuerza
de una señal, aumentando también su amplitud sin cambiar las demás
características de dicha señal.

Para la construcción de un amplificador de señal deben considerarse los
siguientes aspectos:

\begin{itemize}
    \item Calculo de los parámetros del transistor.
    \item Polarización del transistor y calculo del punto de operación en cd.
    \item Analisis en ca.
    \item Analisis de la respuesta en frecuencia del amplificador.
\end{itemize}

Este documento detalla los conceptos necesarios, los cálculos, las mediciones,
las simulaciones y el montaje en una placa de prueba del amplificador de señal
haciendo uso del transistor BJT tipo npn 2N2222A y del transistor JFET canal n
2N3819. Así como el montaje de amplificadores en etapas múltiples.

