\chapter{CAMBIADORES DE CALOR}

\section{Concepto}
Es un equipo térmico en el que un fluido o medio mas ``caliente'' entrega calor
al fluido o medio mas ``frío''.

\section{Objetivos}
\begin{itemize}
    \item Optimización del uso de la energía mediante el aprovechamiento de la
        energía de desecho.
    \item Mejorar el control del proceso de transferencia de calor.
    \item Manejo de grandes cantidades de calor.
\end{itemize}

\section{Tipos de cambiadores}
\begin{itemize}
    \item Cambiador de contacto directo.
    \item Cambiador de contacto de superficie.
    \item Regeneradores.
\end{itemize}

\subsection{Cambiador de contacto directo (CD)}

% grafica ccd

\textbf{Ventajas}
\begin{itemize}
    \item Diseño, calculo y construcción relativamente fáciles.
    \item Costo bajo.
\end{itemize}

\textbf{Desventajas}
\begin{itemize}
    \item Solo intervienen fluidos de la misma naturaleza.
    \item Fluidos con diferentes grados de fuerza, composición, etc.
\end{itemize}

% grafica aplicaciones

\subsection{Cambiador de contacto de superficie (CS)}

% grafica ccs

\textbf{Ventajas}
\begin{itemize}
    \item Interviene cualquier tipo de fluido.
    \item Mayor control de los fluidos participantes.
    \item Aplicaciones industriales.
    \item Manejo de grandes tasas de calor.
\end{itemize}

\textbf{Desventajas}
\begin{itemize}
    \item Calculo y diseño complejos.
    \item Materiales y construcción caros.
    \item Costos de mantenimiento elevados.
\end{itemize}

%\subsection{Regeneradores}
% grafica regenerador

% grafica aplicacion

\section{Cambiadores de contacto de superficie}

\subsection{Cambiador de superficies planas o de flujo transversal}

% grafica ft

\textbf{Ventajas}
\begin{itemize}
    \item Calculo y diseño no complejo.
    \item Materiales y construcción relativamente económicos.
\end{itemize}

\textbf{Desventajas}
\begin{itemize}
    \item Presiones bajas a medias.
    \item Áreas de transferencia de calor limitadas.
\end{itemize}

\subsection{Cambiador de horquillas o de tubos concéntricos}

% grafica tc

\textbf{Ventajas}
\begin{itemize}
    \item Diseño y calculo relativamente fácil.
    \item Materiales y construcción económicos.
\end{itemize}

\textbf{Desventajas}
\begin{itemize}
    \item Área de transferencia de calor reducida.
    \item Gran probabilidad de fugas.
    \item Elevados costos de mantenimiento.
\end{itemize}

\subsection{Cambiador de superficies extendidas}
El objetivo es disminuir las dimensiones del cambiador de calor.

% grafica se

\textbf{Ventajas}
\begin{itemize}
    \item Permite disminuir notablemente las dimensiones del equipo.
\end{itemize}

\textbf{Desventajas}
\begin{itemize}
    \item Diseño y calculo muy complejos.
    \item Material caro adquirido de fabrica.
    \item Mantenimiento caro.
\end{itemize}

\subsection{Cambiador de calor de coraza y tubos}

% grafica cyt

\textbf{Ventajas}
\begin{itemize}
    \item Medianas a elevadas áreas de transferencia de calor.
    \item Permite el manejo de elevadas tasas de calor.
    \item Mayor control del proceso.
    \item Aplicación industrial.
\end{itemize}

\textbf{Desventajas}
\begin{itemize}
    \item Calculo y diseño complicado.
    \item Materiales y fabricación caros.
    \item Mantenimiento caro.
\end{itemize}

\section{Diseño y calculo de cambiadores de calor}

\subsection{Áreas involucradas}
\begin{itemize}
    \item Área térmica.
    \item Área fluida.
    \item Área de materiales.
    \item Área de fabricación.
    \item Área de mantenimiento y operación.
    \item Área económica.
\end{itemize}

\section{Calculo y diseño térmico de un cambiador}

El balance de calor en condiciones adiabáticas, el calor perdido por el fluido
caliente debe ser igual al calor ganado por el fluido frío.

\begin{equation}
    q_h = q_c
\end{equation}

Existen dos tipos de calor a considerarse:

\begin{itemize}
    \item Calor sensible ($q_s$).
    \item Calor latente ($q_l$).
\end{itemize}

\begin{equation*}
    -q_{sh} = q_{sc}
\end{equation*}
\begin{equation}
    -\dot{m}_h\,C_{ph}\,(t_{ho} - t_{hi}) = \dot{m}_c\,C_{pc}\,(t_{co} - t_{ci})
\end{equation}

\begin{equation*}
    -q_{lh} = q_{sc}
\end{equation*}
\begin{equation}
    -\dot{m}_h\,(-\Delta H) = \dot{m}_c\,C_{pc}\,(t_{co} - t_{ci})
\end{equation}

Donde:
\begin{itemize}
    \item $\dot{m}_h$, $\dot{m}_c$: Flujos másicos del fluido caliente y frío
        respectivamente.
    \item $C_{ph}$, $C_{pc}$: Calor especifico de los fluidos caliente y frío
        respectivamente.
    \item $t_{hi}$, $t_{ci}$: Temperaturas de entrada de los fluidos caliente y 
        frío respectivamente.
    \item $t_{ho}$, $t_{co}$: Temperaturas de salida de los fluidos caliente y 
        frío respectivamente.
    \item $\Delta H$: Calor de condensación del fluido caliente.
\end{itemize}

% grafica de cambio de fase

Los objetivos de balance pueden ser:
\begin{itemize}
    \item Obtener el calor.
    \item Obtener un dato flotante.
\end{itemize}

\underline{Ecuación del calor transmitido en el cambiador de calor}:

\begin{equation}
    q = U\,A\,\Delta t
\end{equation}

Donde:
\begin{itemize}
    \item $q$: Calor transmitido.
    \item $U$: Coeficiente global de transferencia de calor.
    \item $A$: Área de transferencia de calor.
    \item $\Delta t$: Diferencia de temperaturas en el cambiador de calor.
\end{itemize}

\begin{equation*}
    -q_h = q_c = q
\end{equation*}
\begin{equation}
    A = \frac{q}{U\,\Delta t}
\end{equation}

\section{Coeficiente global}

\subsection{Pared vertical}

% grafica u pared

\begin{equation*}
    q = \frac{t_i - t_o}{R_{ci} + R_p + R_{co}}
\end{equation*}
\begin{equation*}
    q = \dfrac{t_i - t_o}{\frac{1}{h_i\,A_i} + \frac{\Delta x_p}{k_p\,A_p} + \frac{1}{h_o\,A_o}}
\end{equation*}
\begin{equation*}
    A_i = A_p = A_o = A
\end{equation*}
\begin{equation*}
    q = \dfrac{A\,(t_i - t_o)}{\frac{1}{h_i} + \frac{\Delta x_p}{k_p} + \frac{1}{h_o}}
\end{equation*}

\begin{equation}
    U = \dfrac{1}{\frac{1}{h_i} + \frac{\Delta x_p}{k_p} + \frac{1}{h_o}}
\end{equation}

\subsection{Conductor cilíndrico}

% grafica u cilindrico

\begin{equation*}
    q = \frac{\bar{t}_o - \bar{t}_i}{R_{co} + R_p + R_{ci}}
\end{equation*}
\begin{equation*}
    q = \dfrac{\bar{t}_o - \bar{t}_i}{\frac{1}{h_o\,A_o} + \frac{\Delta r_p}{k_p\,A_p} + \frac{1}{h_i\,A_i}}
\end{equation*}
\begin{equation*}
    A_o \neq A_p \neq A_i
\end{equation*}

Multiplicando por $A_o/A_o$:
\begin{equation*}
    q = \dfrac{A_o(\bar{t}_o - \bar{t}_i)}{\frac{1}{h_o} + \frac{A_o\,\Delta r_p}{k_p\,A_p} + \frac{A_o}{h_i\,A_i}}
\end{equation*}

Considerando:
\begin{equation*}
    A_o = \pi\,D_e\,l
\end{equation*}
\begin{equation*}
    A_i = \pi\,D_i\,l
\end{equation*}
\begin{equation*}
    \frac{A_o}{A_i} = \frac{D_e}{D_i}
\end{equation*}

\begin{equation*}
    \frac{A_o\,\Delta r_p}{A_p} = \frac{\pi\,D_e\,l\,\Delta r_p}{\frac{A_o-A_i}{\ln(\frac{A_o}{A_i})}}
\end{equation*}

\begin{equation*}
    \begin{split}
        A_o - A_i
            &= \pi\,D_e\,l - \pi\,D_i\,l\\
            &= \pi\,l\,(D_e - \,D_i)\\
            &= \pi\,l\,(2r_o - 2r_i)\\
            &= 2\pi\,l\,(r_o - r_i)\\
            &= 2\pi\,l\,\Delta r_p\\
    \end{split}
\end{equation*}

\begin{equation*}
    \begin{split}
        \frac{A_o\,\Delta r_p}{A_p}
            &= \frac{\pi\,D_e\,l\,\Delta r_p}{\frac{2\pi\,l\Delta r_p}{\ln(\frac{A_o}{A_i})}}\\
            &= \frac{D_e}{2}\ln\left(\frac{D_e}{D_i}\right)\\
    \end{split}
\end{equation*}

Por tanto:

\begin{equation*}
    q = \dfrac{A_o(\bar{t}_o - \bar{t}_i)}{\frac{1}{h_o} + \frac{D_e}{2k_p}\ln\left(\frac{D_e}{D_i}\right) + \frac{D_e}{h_i\,D_i}}
\end{equation*}
\begin{equation*}
    q = U_o\,A_o\,\Delta t
\end{equation*}

$U_o$: Coeficiente global referido a la superficie externa.
\begin{equation}
    U_o = \dfrac{1}{\frac{1}{h_o} + \frac{D_e}{2k_p}\ln\left(\frac{D_e}{D_i}\right) + \frac{1}{h_i}\left(\frac{D_e}{D_i}\right)}
\end{equation}

Multiplicando por $A_i/A_i$:
\begin{equation*}
    q = \dfrac{A_i(\bar{t}_o - \bar{t}_i)}{\frac{A_i}{h_o\,A_o} + \frac{A_i\,\Delta r_p}{k_p\,A_p} + \frac{1}{h_i}}
\end{equation*}

Considerando:
\begin{equation*}
    \begin{split}
    \frac{A_i\,\Delta r_p}{A_p}
        &= \frac{\pi\,D_i\,l\,\Delta r_p}{\frac{A_o-A_i}{\ln(\frac{A_o}{A_i})}}\\
            &= \frac{\pi\,D_i\,l\,\Delta r_p}{\frac{2\pi\,l\Delta r_p}{\ln(\frac{A_o}{A_i})}}\\
            &= \frac{D_i}{2}\ln\left(\frac{D_e}{D_i}\right)\\
    \end{split}
\end{equation*}

Por tanto:

\begin{equation*}
    q = \dfrac{A_i(\bar{t}_o - \bar{t}_i)}{\frac{1}{h_i} + \frac{D_i}{2k_p}\ln\left(\frac{D_e}{D_i}\right) + \frac{D_i}{h_o\,D_e}}
\end{equation*}
\begin{equation*}
    q = U_i\,A_i\,\Delta t
\end{equation*}

$U_i$: Coeficiente global referido a la superficie interna.
\begin{equation}
    U_i = \dfrac{1}{\frac{1}{h_i} + \frac{D_i}{2k_p}\ln\left(\frac{D_e}{D_i}\right) + \frac{1}{h_o}\left(\frac{D_i}{D_e}\right)}
\end{equation}

Igualando las expresiones:
\begin{equation*}
    U_o\,A_o\,\Delta t = U_i\,A_i\,\Delta t
\end{equation*}

\begin{equation}
    \frac{U_o}{U_i} = \frac{A_i}{A_o} = \frac{D_i}{D_e}
\end{equation}

\section{Incrustaciones, costras, o ensuciamiento}

% grafica incrustaciones

\begin{equation*}
    q = \frac{\Delta t}{R_{ci} + R_{di} + R_p + R_{do} + R_{co}}
\end{equation*}
\begin{equation*}
    \frac{1}{U_d} = \frac{1}{U} + \sum R_d
\end{equation*}
\begin{equation*}
    \sum R_d = R_{di} + R_{do}
\end{equation*}

\begin{equation}
    \frac{1}{U_{do}} = \frac{1}{U_o} + \sum R_d
\end{equation}
\begin{equation}
    \frac{1}{U_{di}} = \frac{1}{U_i} + \sum R_d
\end{equation}

Donde:
\begin{itemize}
    \item $U_d$: Coeficiente de diseño (con incrustaciones).
    \item $U$: Coeficiente limpio (sin incrustaciones).
\end{itemize}

\begin{equation}
    q = U_d\,A_d\,\Delta t = U_{do}\,A_{do}\,\Delta t = U_{di}\,A_{di}\,\Delta t
\end{equation}

\section{Disposiciones de los fluidos en un cambiador}
Pueden ser:

\begin{itemize}
    \item En contracorriente o contra flujo.
    \item En corriente paralela.
    \item Con cambio de fase.
\end{itemize}

% grafica cambiadores

\subsection{Calculo del gradiente}
El gradiente de temperatura se puede aproximar por medio de:
\begin{equation*}
    \Delta t = \frac{1}{2}\,(\Delta t_1 + \Delta t_2)
\end{equation*}

Para el calculo del valor exacto se procede de la siguiente manera:

% grafica gradiente

\begin{equation*}
    dq = U\,\Delta t\,dA
\end{equation*}

Del gráfico puede verse que:
\begin{equation*}
    \frac{d\Delta t}{dq} = \frac{\Delta t_1 - \Delta t_2}{q}
\end{equation*}

Combinando ambas expresiones:
\begin{equation*}
    \frac{d\Delta t}{U\,\Delta t\,dA} = \frac{\Delta t_1 - \Delta t_2}{q}
\end{equation*}
\begin{equation*}
    \frac{d\Delta t}{U\,\Delta t} = \frac{\Delta t_1 - \Delta t_2}{q}\,dA
\end{equation*}

Se simplifica la expresión considerando que $U$ es constante y no esta en
función de la gradiente de temperatura.
\begin{equation*}
    \int\frac{d\Delta t}{U\,\Delta t} = \int\frac{\Delta t_1 - \Delta t_2}{q}\,dA
\end{equation*}
\begin{equation*}
    \frac{1}{U}\,\ln\left(\frac{\Delta t_1}{\Delta t_2}\right) = \frac{\Delta t_1 - \Delta t_2}{q}\,A
\end{equation*}
\begin{equation*}
    q = U\,A\,\frac{\Delta t_1 - \Delta t_2}{\ln\left(\frac{\Delta t_1}{\Delta t_2}\right)}
\end{equation*}

Por tanto el gradiente logarítmico es:
\begin{equation}
    \Delta t_{log} = \frac{\Delta t_1 - \Delta t_2}{\ln\left(\frac{\Delta t_1}{\Delta t_2}\right)}
\end{equation}

\subsection{Pasos del cambiador}

% grafico cambiadores

\subsection{Factor de corrección ($F_c$)}

\begin{equation*}
    \Delta t = F_c\,\Delta t_{log}
\end{equation*}

Donde:
\begin{itemize}
    \item $\Delta_{log}$ para un cambiador 1:1 en contracorriente.
\end{itemize}

Para cambiadores 1:1 y condensadores el factor de corrección $F_c = 1$.

Para hallar el factor de corrección se usa la tabla de corrección usando los
valores $X$ y $Z$:

\begin{equation}
    X = \frac{t_{c2} - t_{c1}}{t_{h1} - t_{c1}}
\end{equation}
\begin{equation}
    Z = \frac{t_{h1} - t_{h2}}{t_{c2} - t_{c1}}
\end{equation}

\subsection{Eficacia de un cambiador}

\begin{enumerate}
    \item $Z > 1$:
        \begin{equation}
            \eta = \frac{t_{c2} - t_{c1}}{t_{h2} - t_{c1}}
        \end{equation}
    \item $Z < 1$:
        \begin{equation}
            \eta = \frac{t_{h1} - t_{h2}}{t_{h1} - t_{c1}}
        \end{equation}
\end{enumerate}

\section{Calculo del numero de tubos ($N_T$)}

\subsection{Criterio de la transferencia de calor}

\begin{equation*}
    A_o = \pi\,D_e\,L_t
\end{equation*}
\begin{equation*}
    L_t = \frac{A_o}{\pi\,D_e}
\end{equation*}
\begin{equation*}
    N_T = \frac{L_t}{L}
\end{equation*}

Donde:
\begin{itemize}
    \item $L_t$: Longitud de tramo o longitud de $1$ tubo.
\end{itemize}

\subsection{Criterio del flujo másico}

\begin{equation*}
    N_T = \frac{\dot{m}}{\dot{m}_1}
\end{equation*}

Donde:
\begin{itemize}
    \item $\dot{m}$: Flujo másico total (fluido interno).
    \item $\dot{m}_1$: Flujo másico por $1$ tubo.
\end{itemize}

\begin{equation*}
    \dot{m} = v\,A_t\,\rho
\end{equation*}
\begin{equation*}
    \dot{m}_1 = v\,A_{t1}\,\rho
\end{equation*}
\begin{equation*}
    N_T = \frac{A_t}{A_{t1}}
\end{equation*}

\underline{Caso 1:1}:
\begin{equation*}
    N_T\Biggr|_q = N_T\Biggr|_{\dot{m}}
\end{equation*}

\underline{Caso 1:2}:
\begin{equation*}
    N_T\Biggr|_q = 2\,N_T\Biggr|_{\dot{m}}
\end{equation*}

\underline{Caso 1:3}:
\begin{equation*}
    N_T\Biggr|_q = 3\,N_T\Biggr|_{\dot{m}}
\end{equation*}

\section{Flujograma para el calculo de un cambiador de calor}

% grafica flujograma

\section{Ebullición}

% grafica ebullición

\begin{itemize}
    \item $I$: Convección libre.
    \item $II$: Formación de burbujas individuales.
    \item $III$: Formación de burbujas en columnas.
    \item $IV$: Película inestable.
    \item $V$: Película estable.
    \item $VI$: Radiación afecta la película.
\end{itemize}

\section{Condensación}
Es el proceso por el que vapor saturado se convierte en liquido, mediante la
extracción de calor latente.

% grafica condensación

Tipos de condensación:
\begin{enumerate}
    \item Condensación en forma de película.
    \item Condensación en forma de gotas.
\end{enumerate}

Depende de factores como: el tipo de valor o el tipo de superficie.

\begin{equation*}
    q = h_{\text{cond}}\,A\,(t_v - t_s)
\end{equation*}
\begin{equation*}
    q = \dot{m}_v\,(\Delta H)
\end{equation*}

Donde:
\begin{itemize}
    \item $h_{\text{cond}}$: Coeficiente de condensación $[kcal/m^2 h ^\circ C]$.
    \item $A$: Superficie de condensación $[m^2]$.
    \item $t_v$: Temperatura de vapor.
    \item $t_s$: Temperatura de superficie fría.
    \item $\dot{m}_v$: Flujo másico del condensado o del vapor $[kg/h]$.
    \item $\Delta H$: Calor de condensación $[kcal/kg]$.
\end{itemize}

\subsection{Calculo de $h$ de condensación}

\underline{Para superficies verticales}:

Para placas:
\begin{equation}
    h = 1.13\,\left(\frac{k_f^3\,\rho_f^2\,g\,\Delta H}{L\,\mu_f\,(t_v - t_s)}\right)^{\frac{1}{4}}
\end{equation}
Para tubos:
\begin{equation}
    h = 1.18\,\left(\frac{k_f^3\,\rho_f^2\,g\,\pi\,D}{\mu_f\,W}\right)^{\frac{1}{3}}
\end{equation}

\underline{Para tubos horizontales}:
\begin{equation}
    h = 0.725\,\left(\frac{k_f^3\,\rho_f^2\,g\,\Delta H}{N^{2/3}\,D\,\mu_f\,(t_v - t_s)}\right)^{\frac{1}{4}}
\end{equation}

Las propiedades del fluido se calculan a la temperatura media de la película
condensada:

\begin{equation}
    t_f = t_v - \frac{3}{4}\,(t_v - t_s)
\end{equation}

Donde:
\begin{itemize}
    \item $L$: Altura de la superficie.
    \item $D$: Diámetro externo.
    \item $W$: Flujo másico de condensado.
        \begin{equation*}
            W = \frac{\dot{m}_v}{N_T}
        \end{equation*}
    \item $N$: Numero de tubos por columna.
\end{itemize}

\subsection{Calculo de $A$}

\begin{equation*}
    q = h_{\text{cond}}\,A\,(t_v - t_s)
\end{equation*}
\begin{equation*}
    q = \dot{m}_v\,(\Delta H)
\end{equation*}

\begin{equation}
    A = \frac{\dot{m}_v\,(\Delta H)}{h_{\text{cond}}\,A\,(t_v - t_s)}
\end{equation}

\section{Método NUT (Numero de unidades térmicas)}
Su importancia radica en que no es necesario conocer las temperaturas de salida
de los fluidos.

\begin{equation}
    \epsilon = \frac{q_r}{q_{\text{max}}}
\end{equation}
\begin{equation*}
    q_{\text{max}} = C_{\text{min}}\,(t_{h1} - t_{c1})
\end{equation*}
\begin{equation*}
    C_{\text{min}} = \min(C_h, C_c)
\end{equation*}

\begin{equation*}
    C_h = \dot{m}_h\,C_{ph}
\end{equation*}
\begin{equation*}
    C_c = \dot{m}_c\,C_{pc}
\end{equation*}

Donde:
\begin{itemize}
    \item $q_r$: Calor real transmitido.
        \begin{equation*}
            q_r = -q_h = q_c
        \end{equation*}
    \item $q_{\text{max}}$: Calor máximo transmitido en un cambiador de calor
        1:1 de área infinita.
    \item $C$: Capacidad térmica.
        \begin{equation*}
            C = \dot{m}\,C_p
        \end{equation*}
\end{itemize}

\begin{equation}
    \text{NUT} = \frac{U\,A}{C_{\text{min}}}
\end{equation}

Físicamente en NUT nos da la idea del tamaño del cambiador.

\section{Superficies aletadas, aleteadas o extendidas}

% grafica aleta

\underline{Balance}:
\begin{equation*}
    q\Biggr|_x = q\Biggr|_{x+\Delta x} + q_c
\end{equation*}

Donde:
\begin{itemize}
    \item $q\Biggr|_x$: Calor que ingresa el VC.
    \item $q\Biggr|_{x+\Delta x}$: Calor que sale del VC.
    \item $q_c$: Calor que se transfiere por convección del VC al medio exterior.
    \item VC: Volumen de control.
\end{itemize}

\begin{equation*}
    q\Biggr|_x - q\Biggr|_{x+\Delta x} - q_c = 0
\end{equation*}
\begin{equation*}
    -k_x\,A(x)\,\frac{dt}{dx}\Biggr|_x - \left(-k_x\,A(x)\,\frac{dt}{dx}\Biggr|_{x + \Delta x}\right) - h\,S(x)\,(t-t_{\infty}) = 0
\end{equation*}
\begin{equation*}
    S(x) = P(x)\,A(x)
\end{equation*}
\begin{equation*}
    -k_x\,A(x)\,\frac{dt}{dx}\Biggr|_x + k_x\,A(x)\,\frac{dt}{dx}\Biggr|_{x + \Delta x} - h\,P(x)\,A(x)\,(t-t_{\infty}) = 0
\end{equation*}

\textbf{Ecuación general de una superficie extendida}:
\begin{equation}
    \frac{d}{dx}\left[k_x\,A(x)\,\frac{dt}{dx}\right] - h\,P(x)\,(t-t_{\infty}) = 0
\end{equation}

Donde:
\begin{itemize}
    \item $A(x)$: Área transversal de la aleta.
    \item $P(x)$: Perímetro de la aleta.
    \item $t$: Temperatura de la aleta.
    \item $t_{\infty}$: Temperatura del medio exterior.
    \item $h$: Coeficiente de convección de la aleta.
\end{itemize}

\subsection{Caso: Superficie con sección transversal uniformemente variable}

% grafica aleta1

\begin{equation}
    \frac{d}{dx}\left[\left[A_o + (A_L - A_o)\,\frac{x}{L}\right]\,\frac{dt}{dx}\right] - \frac{h}{k}\left[P_o + (P_L - P_o)\,\frac{x}{L}\right]\,(t - t_{\infty}) = 0
\end{equation}

Donde:
\begin{itemize}
    \item $A_o$: Área en la base.
    \item $P_o$: Perímetro en la base.
    \item $A_L$: Área en el extremo.
    \item $P_L$: Perímetro en el extremo.
\end{itemize}

\subsection{Caso: Aletas de sección transversal uniforme}

% grafica aleta2

\begin{equation}
    \frac{d^2 t}{dx^2} - \frac{h\,P}{k\,A}\,(t - t_{\infty}) = 0
\end{equation}

\subsection{Caso: Aletas circulares de espesor constante}

% grafica aleta3

\begin{equation*}
    \frac{d}{dr}\left(r\,\frac{dt}{dr}\right)-\frac{h\,r}{k\,t}\,(t-t_{\infty}) = 0
\end{equation*}

Soluciones para aletas de sección transversal constante:
\begin{equation*}
    \frac{d^2 t}{dx^2} - \frac{h\,P}{k\,A}\,(t - t_{\infty}) = 0
\end{equation*}
\begin{equation*}
    \theta = t - t_{\infty}
\end{equation*}
\begin{equation*}
    m^2 = \frac{h\,P}{k\,A}
\end{equation*}
\begin{equation}
    \frac{d^2 t}{dx^2} - m^2\,\theta = 0
\end{equation}

\underline{Solución general}:
\begin{equation*}
    \theta = C_1\,e^{mx} + C_2\,e^{-mx}
\end{equation*}
\begin{equation*}
    \theta = A\,\cosh(mx) + B\,\senh(mx)
\end{equation*}

\underline{Solución particular}:
Se debe resolver los coeficientes con las condiciones de contorno.

\begin{enumerate}
    \item \textbf{Aletas muy largas}:
        \begin{equation*}
            x = 0 \rightarrow \theta = \theta_0
        \end{equation*}
        \begin{equation*}
            x = \infty \rightarrow \theta = \theta
        \end{equation*}
        \begin{equation*}
            \frac{\theta}{\theta_0} = e^{-mx}
        \end{equation*}
        \begin{equation}
            \frac{t - t_{\infty}}{t_i - t_{\infty}} = e^{-mx}
        \end{equation}
    \item \textbf{Temperatura conducida en $x = L$}:
        \begin{equation*}
            x = 0 \rightarrow \theta = \theta_0
        \end{equation*}
        \begin{equation*}
            x = L \rightarrow \theta = \theta_L
        \end{equation*}
        \begin{equation*}
            \theta_L = t_L - t_{\infty}
        \end{equation*}
        \begin{equation*}
            \theta_0 = t_0 - t_{\infty}
        \end{equation*}
        \begin{equation*}
            \theta = t - t_{\infty}
        \end{equation*}
        \begin{equation}
            \frac{\theta}{\theta_0} = \frac{t - t_{\infty}}{t_0 - t_{\infty}} = \left(\frac{\theta_L}{\theta_0} - e^{-mL}\right)\left(\frac{e^{mx} - e^{-mx}}{e^{mL} - e^{-mL}}\right) + e^{-mx}
        \end{equation}
    \item \textbf{Extremo de la aleta aislado}:
        \begin{equation*}
            x = 0 \rightarrow \theta = \theta_0
        \end{equation*}
        \begin{equation*}
            x = L \rightarrow \frac{d\theta}{dx} = \theta_0
        \end{equation*}
        \begin{equation}
            \frac{\theta}{\theta_0} = \frac{t - t_{\infty}}{t_0 - t_{\infty}} = \frac{e^{mx}}{1 + e^{2mL}} + \frac{e^{-mx}}{1 + e^{-2mL}}
        \end{equation}
    \item \textbf{Conducción en el extremo es igual a la convección en el extremo}:
        \begin{equation*}
            x = 0 \rightarrow \theta = \theta_0
        \end{equation*}
        \begin{equation*}
            x = L \rightarrow k\,\frac{d\theta}{dx} = h\,\theta
        \end{equation*}
        \begin{equation}
            \frac{\theta}{\theta_0} = \dfrac{t - t_{\infty}}{t_0 - t_{\infty}} = \frac{\cosh(m(L-x))+(\frac{h}{m\,k})\senh(m(L-x))}{\cosh(mL)+(\frac{h}{m\,k})\senh(mL)}
        \end{equation}
\end{enumerate}

\subsection{Método de la eficiencia de aleta}

% grafica aleta4

\begin{equation}
    \frac{\theta}{\theta_0} = \dfrac{t - t_{\infty}}{t_0 - t_{\infty}} = \frac{I_o(n\,r)\,K(n\,r_0)+K_0(n\,r)\,I_1(n\,r_2)}{I_0(n\,r_1)\,K_1(n\,r_2)+K_0(n\,r)\,I_1(n\,r_2)}
\end{equation}
\begin{equation*}
    n = \sqrt{\frac{2h}{kr}}
\end{equation*}

Donde:
\begin{itemize}
    \item $I_o$: Ecuación de \emph{Bessel} modificada de primera especie.
    \item $K_1$: Ecuación de \emph{Bessel} modificada de segunda especie.
\end{itemize}

\underline{Balance}:
\begin{equation*}
    q_t = q_a + q_{s/a}
\end{equation*}
\begin{equation*}
    q_a = \eta_a\,h_a\,A_a\,(t - t_{\infty})
\end{equation*}
\begin{equation*}
    q_{s/a} = h_o\,A_{s/a}\,(t - t_{\infty})
\end{equation*}
\begin{equation*}
    q_t = \eta_a\,h_a\,A_a\,(t - t_{\infty}) + h_o\,A_{s/a}\,(t - t_{\infty})
\end{equation*}

\begin{equation*}
    A_{s/a} = N_a\,A_{s/a1}
\end{equation*}
\begin{equation*}
    A_{a} = N_a\,A_{a1}
\end{equation*}

\begin{equation*}
    q = N_a\,[h_o\,A_{s/a1}\,(t_o - t_{\infty} + \eta_a\,h_a\,A_{a1}\,(t - t_{\infty}]
\end{equation*}

Donde:
\begin{itemize}
    \item $N_a$: Numero de aletas.
    \item $A_{s/a1}$: Área libre entre dos aletas.
    \item $A_{a1}$: Área de una aleta.
\end{itemize}

\underline{Simplificación}:
\begin{equation*}
    h_o \approx h_a
\end{equation*}

\begin{equation*}
    q = N_a\,h_o\,(t_o - t_{\infty})(A_{s/a1} + \eta\,A_{a1})
\end{equation*}

