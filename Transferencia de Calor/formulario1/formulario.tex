\documentclass[letter,oneside,10pt]{article}

\usepackage[spanish,es-nodecimaldot]{babel}
\usepackage[utf8]{inputenc}

\usepackage{helvet}
\renewcommand{\familydefault}{\sfdefault}
\usepackage[T1]{fontenc}
\usepackage{textcomp}

\usepackage{adjustbox}

\usepackage{amsmath}
\usepackage{amssymb}
\usepackage{siunitx}
\usepackage{multicol}

\usepackage{anysize}
\marginsize{2.5cm}{2cm}{1cm}{3cm}

\setlength{\parskip}{0pt}
\setlength{\parindent}{0pt}
\setlength\abovedisplayskip{0pt}
\setlength\belowdisplayskip{0pt}
\special{papersize=215.9mm,279.4mm}

\usepackage[nodisplayskipstretch]{setspace}
\setstretch{0.1}

\usepackage{float}
\usepackage{fancyhdr}
\usepackage[Sonny]{fncychap}
\usepackage{lastpage}
\pagestyle{fancy}
\fancyhf{}
\fancyhead[RE]{\leftmark}
\fancyhead[LO]{}
\fancyfoot[CO,CE]{\thepage}

\usepackage{titlesec}
\titleformat{\section}
{\normalfont\fontsize{6}{10}\bfseries}{\thesection}{1em}{}

\usepackage[
    pdfauthor={Carlos Eduardo Caballero Burgoa},%
    pdftitle={Transferencia de Calor},%
    pdfsubject={Formulario},%
    colorlinks,%
    citecolor=black,%
    filecolor=black,%
    linkcolor=black,%
    urlcolor=black,
    breaklinks]{hyperref}
\usepackage{breakurl}

\newcommand{\blankpage}{
\newpage
\thispagestyle{empty}
\mbox{}
\newpage
}

\DeclareMathOperator{\fer}{fer}

\begin{document}

\thispagestyle{empty}
\begin{multicols}{3}

\scriptsize
\section*{FORMULAS ÚTILES}
%\vspace{-0.3cm}
Caída libre:
\vspace{-0.3cm}
\begin{equation*}
    V_f^2 - V_i^2 = 2gh
\end{equation*}
Área de un circulo:
\vspace{-0.3cm}
\begin{equation*}
    A = \pi\,r^2
\end{equation*}
Área de una esfera:
\vspace{-0.3cm}
\begin{equation*}
    A = 4\pi R^3
\end{equation*}
Volumen de una esfera:
\vspace{-0.3cm}
\begin{equation*}
    V = \frac{4}{3}\pi R^3
\end{equation*}

%\vspace{-0.5cm}
\section*{CONDUCCIÓN PERMANENTE}
%\vspace{-0.5cm}
\begin{equation*}
    \begin{split}
        q &= k\,A\,\frac{\Delta T}{\Delta x}\\
        q &= \frac{\Delta T}{R_T}\\
        R_T &= \frac{\Delta X}{k_x\,A_x}\\
        q &= m\,C_p\,\frac{dt}{d\theta}\\
    \end{split}
\end{equation*}
%\vspace{-0.2cm}
Área cilindro:
\vspace{-0.4cm}
\begin{equation*}
    A=\frac{A_E-A_I}{\ln\left(\frac{A_E}{A_I}\right)}
\end{equation*}
%\vspace{-0.2cm}
Área esfera:
\vspace{-0.4cm}
\begin{equation*}
    A=\sqrt{A_E\,A_I}
\end{equation*}
\underline{Costo óptimo}:
%\vspace{-0.2cm}
\begin{equation*}
    \begin{split}
        C_{\text{T}} &= C_{\text{F}}+C_{\text{V}}\\
        C_{\text{F}} &= nA_x[\text{m}^2]\,\text{CA}
                        \left[\frac{\text{Bs}}{\text{m}^2}\right]
                        \frac{1}{\text{TV}[\text{año}]}\\
        C_{\text{V}} &= \frac{t_f-t_i}{R_p+nR_x}
                        \left[\frac{\text{kcal}}{\text{h}}\right]
                        \frac{1[\text{kg}]}{\text{PC}[\text{kcal}]}\cdots\\
                     &  \cdots\text{CC}\left[\frac{\text{Bs}}{\text{kg}}\right]
                        \text{FU}\left[\frac{\text{h}}{\text{año}}\right]\\
    \end{split}
\end{equation*}
%\vspace{-0.3cm}
\tiny
\begin{itemize}
    \item CA: Costo del aislante.
    \item TV: Tiempo de vida.
    \item PC: Poder calorífico.
    \item CC: Costo de combustible.
    \item FU: Frecuencia de uso.
\end{itemize}

\scriptsize
%\vspace{-0.5cm}
\section*{CONDUCCIÓN TRANSITORIA}
%\vspace{-0.5cm}
\begin{equation*}
    \text{Bi} = \frac{h\,L}{k};\quad
    \text{Fo} = \frac{\alpha\,\theta}{L^2};\quad
    L = \frac{V}{A};\quad
    \alpha = \frac{k_x}{\rho\,C_p}
\end{equation*}
%\vspace{-0.3cm}
\underline{Caso}: $\text{Bi}\leq 0.1$
\begin{equation*}
    \begin{split}
        \theta &= -\frac{m\,C_p}{h\,A}\,\ln
                  \left(\frac{t_f-t_{\infty}}{t_i-t_{\infty}}\right)\\
        \frac{t-t_{\infty}}{t_i-t_{\infty}} &= e^{-\text{Bi}\,\text{Fo}}\\
    \end{split}
\end{equation*}
\underline{Caso}: $\text{Bi}>0.1$\\
Método analítico especial\\
(para $A\rightarrow\infty$ o $h\rightarrow\infty$):
%\vspace{-0.3cm}
\begin{equation*}
    \begin{split}
        \frac{t_f-t_\infty}{t_i-t_\infty}
            & = \frac{4}{\pi}\biggr[e^{-{a_1}X}\,
            \sen\left(\frac{\pi{x}}{2{r_m}}\right)+\cdots\\
            & \cdots+\frac{1}{3}\,e^{-9{a_1}X}\,
            \sen\left(\frac{3\pi{x}}{2{r_m}}\right)+
            \cdots\biggr]\\
        a_1 &= \left(\frac{\pi}{2}\right)^2\\
        X &= \frac{\alpha\,\theta}{r_m^2}\\
    \end{split}
\end{equation*}
\underline{Método analítico-gráfico}:\\
Temperatura relativa:
\vspace{-0.3cm}
\begin{equation*}
    y = \frac{t_f-t_\infty}{t_i-t_\infty}
\end{equation*}
%\vspace{-0.3cm}
Tiempo relativo:
\vspace{-0.5cm}
\begin{equation*}
    x = \frac{\alpha\,\theta}{r_m^2}
\end{equation*}
%\vspace{-0.3cm}
Resistencia relativa:
\vspace{-0.4cm}
\begin{equation*}
    \quad\quad\quad\quad m = \frac{k}{h\,r_m}
\end{equation*}
%\vspace{-0.3cm}
Posición relativa:
\vspace{-0.4cm}
\begin{equation*}
    \quad n = \frac{r}{r_m}
\end{equation*}
%\vspace{-0.3cm}
\underline{Método gráfico}:
%\vspace{-0.4cm}
\begin{equation*}
    \frac{\Delta x^2}{\alpha\,\Delta\theta}=M
\end{equation*}
%\vspace{-0.3cm}
\tiny
\begin{itemize}
    \item $\Delta\theta$: Incremento de tiempo.
    \item $M=2$: para flujo en una dimensión.
    \item $M=4$: para flujo en dos dimensiones.
    \item $M=6$: para flujo en tres dimensiones.
\end{itemize}
%\vspace{-0.2cm}
\scriptsize
\begin{equation*}
    N_{\Delta\theta}=\frac{\theta}{\Delta\theta}
\end{equation*}
%\vspace{-0.3cm}
\tiny
\begin{itemize}
    \item $N_{\Delta\theta}$: Número de incrementos de $\theta$.
    \item $\theta$: Tiempo de proceso.
\end{itemize}
\scriptsize
\underline{Cuerpo semi-infinito}:\\
\underline{Caso}: $h=\infty$
%\vspace{-0.3cm}
\begin{equation*}
    \frac{t-t_\infty}{t_i-t_\infty} = \fer\left(
        \frac{y}{\sqrt{4\alpha\theta}}
    \right)
\end{equation*}
%\vspace{-0.3cm}
\tiny
\begin{itemize}
    \item $y$: Profundidad del plano.
    \item $t_i$: Temperatura del suelo.
    \item $t_\infty$: Temperatura del medio fluido (exterior).
    \item $t_s$: Temperatura de la superficie del suelo ($t_s = t_\infty$).
    \item $\fer$: Función error.
\end{itemize}
\scriptsize
\underline{Caso}: $h\ll\infty$
%\vspace{-0.3cm}
\begin{equation*}
    \begin{split}
        \frac{t-t_i}{t_\infty-t_i} = 
        1 -
        \text{fer}(\xi) -
        \Biggl\{
            e^{\left(
                \frac{h\,y}{k}+\frac{h^2\alpha\theta}{k^2}
            \right)}
        \cdots\\
        \cdots\left[
            1 - \text{fer}\left(
                \xi + \frac{h\sqrt{\alpha\theta}}{k}
            \right)
        \right]
        \Biggl\}
    \end{split}
\end{equation*}
%\vspace{-0.3cm}
\begin{equation*}
    \xi = \frac{y}{\sqrt{4\alpha\theta}}
\end{equation*}

%\vspace{-0.5cm}
\section*{CONVECCIÓN NATURAL}
%\vspace{-0.3cm}
\begin{equation*}
    q_c = h\,A\,(t_s-t_\infty)
\end{equation*}
%\vspace{-0.3cm}
\underline{Ecuaciones de \emph{Rice}} ($\text{Gr} > 3$):
\begin{equation*}
    \text{Nu}_f = 0.47\,(\text{Gr}_f\,\text{Pr}_f)^{0.25}
    \qquad\text{Tubos horizontales}
\end{equation*}
%\vspace{-0.3cm}
\begin{equation*}
    \text{Nu}_f = 0.59\,(\text{Gr}_f\,\text{Pr}_f)^{0.25}
    \qquad\text{Tubos verticales}
\end{equation*}
%\vspace{-0.1cm}
\begin{equation*}
    \text{Pr} = C_p\left(\frac{\mu}{k}\right)
\end{equation*}
%\vspace{-0.1cm}
\begin{equation*}
    \text{Gr} = \frac{g\,D^3\,\beta\,\Delta t}{\gamma^2}
\end{equation*}
\tiny
\begin{itemize}
    \item $D$: Longitud característica.
        \begin{itemize}
            \item Para tubos horizontales: $D=D_E$ (Diámetro externo).
            \item Para tubos verticales: $D=L$ (Altura).
        \end{itemize}
\end{itemize}
\scriptsize
\begin{equation*}
    \text{Nu} = \frac{h\,D}{k}
\end{equation*}
%\vspace{-0.5cm}
\tiny
\begin{itemize}
    \item $D$: Longitud característica.
        \begin{itemize}
            \item Para ambos tubos: $D=D_E$ (Diámetro externo).
        \end{itemize}
\end{itemize}
\scriptsize
\underline{Caso}: Aire (flujo laminar)
%\vspace{-0.3cm}
\begin{equation*}
    \begin{split}
        h &= 2.1\,\Delta t^{0.25}\qquad\text{Paredes horizontales hacia arriba}\\
        h &= 1.1\,\Delta t^{0.25}\qquad\text{Paredes horizontales hacia abajo}\\
        h &= 1.5\,\Delta t^{0.25}\qquad\text{Paredes verticales}\,(L > 0.40)\\
        h &= 1.2\,\left(\frac{\Delta t}{L}\right)^{0.25}\qquad\text{Paredes verticales}\,(L < 0.40)\\
        h &= 1.1\,\left(\frac{\Delta t}{D}\right)^{0.25}\qquad\text{Tubos horizontales y verticales}\\
    \end{split}
\end{equation*}
\underline{Caso}: Otros fluidos, flujo turbulento, otras formas geométricas:
%\vspace{-0.3cm}
\begin{equation*}
    \text{Nu} = C\,Ra^a
\end{equation*}
%\vspace{-0.3cm}
\underline{Convección transitoria}:
\begin{equation*}
    \begin{split}
        -q_c &= q_s\\
        -h\,A\,(t_s-t) &= m\,C_p\,\left(\frac{dt}{d\theta}\right)\\
        \theta &= -\frac{m\,\bar{C_p}}{A\,\bar{h}}\,\ln\left(\frac{t_s-f_f}{t_s-t_i}\right)\\
    \end{split}
\end{equation*}
\underline{Casos particulares}:
Múltiples tubos horizontales y/o verticales:
%\vspace{-0.3cm}
\begin{equation*}
    \begin{split}
        A_T &= N_t A_t\\
        A_t &= \pi D_E\,l\\
\end{split}
\end{equation*}
En placas verticales se debe tomar el volumen por encima de la placa vertical.
\\

\underline{Cavidades}\\
$(L/b)>3$:
%\vspace{-0.3cm}
\begin{equation*}
    \begin{split}
        \bar{t} &= \frac{T_1 + T_2}{2}\\
        \text{Gr} &= \frac{g\,b^3\,\beta\,(T_1 - T_2)}{\gamma^2}\\
    \end{split}
\end{equation*}
Conducción pura\\
($\text{Gr}<2000$):
%\vspace{-0.3cm}
\begin{equation*}
    \text{Nu} = 1
\end{equation*}
Convección natural en régimen laminar\\
($\num{2e4} < \text{Gr} < \num{2e5}$):
%\vspace{-0.3cm}
\begin{equation*}
    \text{Nu} = 0.18\,\text{Gr}^{\frac{1}{4}}\,
    \left(\frac{L}{b}\right)^{-\frac{1}{9}}
\end{equation*}
Convección natural en régimen turbulento\\
($\num{2e5} < \text{Gr} < \num{2e7}$):
%\vspace{-0.3cm}
\begin{equation*}
    \text{Nu} = 0.065\,\text{Gr}^{\frac{1}{3}}\,
    \left(\frac{L}{b}\right)^{-\frac{1}{9}}
\end{equation*}

\end{multicols}

\end{document}

