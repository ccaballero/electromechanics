\documentclass[letter,oneside,10pt]{article}

\usepackage[spanish,es-nodecimaldot]{babel}
\usepackage[utf8]{inputenc}

\usepackage{helvet}
\renewcommand{\familydefault}{\sfdefault}
\usepackage[T1]{fontenc}
\usepackage{textcomp}

\usepackage{adjustbox}

\usepackage{amsmath}
\usepackage{amssymb}
\usepackage{siunitx}
\usepackage{multicol}

\usepackage{anysize}
\marginsize{2.5cm}{2cm}{1cm}{3cm}

\setlength{\parskip}{0pt}
\setlength{\parindent}{0pt}
\setlength\abovedisplayskip{0pt}
\setlength\belowdisplayskip{0pt}
\special{papersize=215.9mm,279.4mm}

\usepackage[nodisplayskipstretch]{setspace}
\setstretch{0.1}

\usepackage{float}
\usepackage{fancyhdr}
\usepackage[Sonny]{fncychap}
\usepackage{lastpage}
\pagestyle{fancy}
\fancyhf{}
\fancyhead[RE]{\leftmark}
\fancyhead[LO]{}
\fancyfoot[CO,CE]{\thepage}

\usepackage{titlesec}
\titleformat{\section}
{\normalfont\fontsize{6}{10}\bfseries}{\thesection}{1em}{}

\usepackage[
    pdfauthor={Carlos Eduardo Caballero Burgoa},%
    pdftitle={Transferencia de Calor},%
    pdfsubject={Formulario},%
    colorlinks,%
    citecolor=black,%
    filecolor=black,%
    linkcolor=black,%
    urlcolor=black,
    breaklinks]{hyperref}
\usepackage{breakurl}

\newcommand{\blankpage}{
\newpage
\thispagestyle{empty}
\mbox{}
\newpage
}

\DeclareMathOperator{\fer}{fer}

\begin{document}

\begin{multicols}{3}

\scriptsize
\section*{FORMULAS ÚTILES}
\vspace{-0.3cm}
Caída libre:
\vspace{-0.3cm}
\begin{equation*}
    V_f^2 - V_o^2 = 2gh
\end{equation*}
Área de un circulo:
\vspace{-0.3cm}
\begin{equation*}
    A = \pi\,r^2
\end{equation*}
Área de una esfera:
\vspace{-0.3cm}
\begin{equation*}
    A = 4\pi R^3
\end{equation*}
Volumen de una esfera:
\vspace{-0.3cm}
\begin{equation*}
    V = \frac{4}{3}\pi R^3
\end{equation*}

\vspace{-0.5cm}
\section*{CONDUCCIÓN PERMANENTE}
\vspace{-0.5cm}
\begin{equation*}
    q = k\,A\,\frac{\Delta T}{\Delta x};\quad
    q = \frac{\Delta T}{R_T};\quad
    R_T = \frac{\Delta x}{k_x\,A_x}
\end{equation*}
\vspace{-0.3cm}
\begin{equation*}
    q = m\,C_p\,\frac{dt}{d\theta}
\end{equation*}
\vspace{-0.2cm}
Área cilindro:
\vspace{-0.4cm}
\begin{equation*}
    A=\frac{A_e-A_i}{\ln\left(\frac{A_e}{A_i}\right)}
\end{equation*}
\vspace{-0.2cm}
Área esfera:
\vspace{-0.4cm}
\begin{equation*}
    A=\sqrt{A_i\,A_e}
\end{equation*}
\underline{Costo óptimo}:
\vspace{-0.3cm}
\begin{equation*}
    C_{\text{t}}=C_{\text{f}}+C_{\text{v}}
\end{equation*}
\begin{equation*}
    C_{\text{f}}=nA_i[\text{m}^2]
    CA\left[\frac{\text{Bs}}{\text{m}^2}\right]
    \frac{1}{TV[\text{año}]}
\end{equation*}
\begin{equation*}
    \begin{split}
        C_{\text{v}}=\frac{t_i-t_o}{R_p+nR_i}
        \left[\frac{\text{kcal}}{\text{h}}\right]
        \frac{1[\text{kg}]}{PC[\text{kcal}]}\\
        CC\left[\frac{\text{Bs}}{\text{kg}}\right]
        FU\left[\frac{\text{h}}{\text{año}}\right]
    \end{split}
\end{equation*}
\vspace{-0.3cm}
\tiny
\begin{itemize}
    \item $CA$: Costo del aislante.
    \item $TV$: Tiempo de vida.
    \item $PC$: Poder calorífico.
    \item $CC$: Costo de combustible.
    \item $FU$: Frecuencia de uso.
\end{itemize}

\scriptsize
\vspace{-0.5cm}
\section*{CONDUCCIÓN TRANSITORIA}
\vspace{-0.5cm}
\begin{equation*}
    \text{Bi} = \frac{h\,L}{k};\quad
    \text{Fo} = \frac{\alpha\,\theta}{L^2};\quad
    L = \frac{V}{A};\quad
    \alpha = \frac{k_x}{\rho\,C_p}
\end{equation*}
\vspace{-0.3cm}
\underline{Caso}: $Bi\leq 0.1$
\begin{equation*}
    \theta = f(\theta) =
    -\frac{m\,C_p}{h\,A}\,\ln\left(\frac{t_f-t_{\infty}}{t_i-t_{\infty}}\right)
\end{equation*}
\vspace{-0.3cm}
\begin{equation*}
    \frac{t-t_{\infty}}{t_i-t_{\infty}} = e^{-Bi\,Fo}
\end{equation*}
\underline{Caso}: $Bi>0.1$. método analítico especial, para $A\rightarrow\infty$
o $h\rightarrow\infty$:
\vspace{-0.3cm}
\begin{equation*}
    \begin{split}
        \frac{t_f-t_\infty}{t_i-t_\infty}
            & = \frac{4}{\pi}[e^{-a_1 X}\,\sen\left(\frac{\pi x}{2r_m}\right)+\\
            & \frac{1}{3}\,e^{-9a_1 X}\,\sen\left(\frac{3\pi x}{2r_m}\right)+
            \cdots]
    \end{split}
\end{equation*}
\vspace{-0.3cm}
\begin{equation*}
    a_1 = \left(\frac{\pi}{2}\right)^2
\end{equation*}
\vspace{-0.3cm}
\begin{equation*}
    X = \frac{\alpha\,\theta}{r_m^2}
\end{equation*}
\underline{Método analítico-gráfico}:\\
Temperatura relativa:
\vspace{-0.3cm}
\begin{equation*}
    y = \frac{t_f-t_\infty}{t_i-t_\infty}
\end{equation*}
\vspace{-0.3cm}
Tiempo relativo:
\vspace{-0.5cm}
\begin{equation*}
    x = \frac{\alpha\theta}{r_m^2}
\end{equation*}
\vspace{-0.3cm}
Resistencia relativa:
\vspace{-0.4cm}
\begin{equation*}
    \quad\quad\quad\quad m = \frac{k}{h\,r_m}
\end{equation*}
\vspace{-0.3cm}
Posición relativa:
\vspace{-0.4cm}
\begin{equation*}
    \quad n = \frac{r}{r_m}
\end{equation*}
\vspace{-0.3cm}
\underline{Método gráfico}:
\vspace{-0.4cm}
\begin{equation*}
    \frac{\Delta x^2}{\alpha\Delta\theta}=M
\end{equation*}
\vspace{-0.3cm}
\tiny
\begin{itemize}
    \item $\Delta\theta$: Incremento de tiempo.
    \item $M=2$ para flujo en una dimensión.
    \item $M=4$ para flujo en dos dimensiones.
    \item $M=6$ para flujo en tres dimensiones.
\end{itemize}
\vspace{-0.2cm}
\scriptsize
\begin{equation*}
    N_{\Delta\theta}=\frac{\theta}{\Delta\theta}
\end{equation*}
\vspace{-0.3cm}
\tiny
\begin{itemize}
    \item $N_{\Delta\theta}$: Número de incrementos de $\theta$.
    \item $\theta$: Tiempo de proceso.
\end{itemize}
\scriptsize
\underline{Cuerpo semi-infinito}:\\
\underline{Caso}: $h=\infty$
\vspace{-0.3cm}
\begin{equation*}
    \frac{t-t_\infty}{t_i-t_\infty} = \fer\left(
        \frac{y}{\sqrt{4\alpha\theta}}
    \right)
\end{equation*}
\vspace{-0.3cm}
\tiny
\begin{itemize}
    \item $y$: Profundidad del plano.
    \item $t_i$: Temperatura del suelo.
    \item $t_\infty$: Temperatura del medio fluido (exterior).
    \item $t_s$: Temperatura de la superficie del suelo.
    \item $\fer$: Función error.
\end{itemize}
\scriptsize
\underline{Caso}: $h\ll\infty$
\vspace{-0.3cm}
\begin{equation*}
    \begin{split}
        \frac{t-t_i}{t_\infty-t_i} = 
        1 -
        \text{fer}(\xi) -
        \left[
            e^{\left(
                \frac{h\,y}{k}+\frac{h^2\alpha\theta}{k^2}
            \right)}
        \right]\\
        \left[
            1 - \text{fer}\left(
                \xi + \frac{h\sqrt{\alpha\theta}}{k}
            \right)
        \right]
    \end{split}
\end{equation*}
\vspace{-0.3cm}
\begin{equation*}
    \xi = \frac{y}{\sqrt{4\alpha\theta}}
\end{equation*}

\vspace{-0.5cm}
\section*{CONVECCIÓN NATURAL}
\vspace{-0.3cm}
\begin{equation*}
    q_c = h\,A\,(t_s-t_\infty)
\end{equation*}
\vspace{-0.3cm}
\underline{Ecuaciones de \emph{Rice}} ($\text{Gr} > 3$):
\begin{equation*}
    \text{Nu}_f = 0.47\,(\text{Gr}_f\,\text{Pr}_f)^{0.25}
    \qquad\text{Tubos hor.}
\end{equation*}
\vspace{-0.3cm}
\begin{equation*}
    \text{Nu}_f = 0.59\,(\text{Gr}_f\,\text{Pr}_f)^{0.25}
    \qquad\text{Tubos ver.}
\end{equation*}
\vspace{-0.1cm}
\begin{equation*}
    \text{Pr} = C_p\,\frac{\mu}{k}
\end{equation*}
\vspace{-0.3cm}
\begin{equation*}
    \text{Gr} = \frac{g\,D^3\,\beta\,\Delta t}{\gamma^2}
\end{equation*}
\vspace{-0.3cm}
\tiny
\begin{itemize}
    \item $D$: Longitud característica.
        \begin{itemize}
            \item Para tubos horizontales: $D=D_E$ (Diámetro externo).
            \item Para tubos verticales: $D=L$ (Altura).
        \end{itemize}
\end{itemize}
\scriptsize
\vspace{-0.1cm}
\begin{equation*}
    \text{Nu} = \frac{h\,D}{k}
\end{equation*}
\vspace{-0.3cm}
\tiny
\begin{itemize}
    \item $D$: Longitud característica.
        \begin{itemize}
            \item Para ambos tubos: $D=D_E$ (Diámetro externo).
        \end{itemize}
\end{itemize}
\scriptsize
\underline{Caso}: Aire (flujo laminar)
\vspace{-0.3cm}
\begin{equation*}
    h = 2.1\,\Delta t^{0.25}\qquad\text{Paredes hor.\ hacia arriba}
\end{equation*}
\begin{equation*}
    h = 1.1\,\Delta t^{0.25}\qquad\text{Paredes hor.\ hacia abajo}
\end{equation*}
\begin{equation*}
    h = 1.5\,\Delta t^{0.25}\qquad\text{Paredes vert.} L > 0.40
\end{equation*}
\begin{equation*}
    h = 1.2\,\left(\frac{\Delta t}{L}\right)^{0.25}
    \qquad\text{Paredes ver.} L < 0.40
\end{equation*}
\begin{equation*}
    h = 1.1\,\left(\frac{\Delta t}{D}\right)^{0.25}
    \qquad\text{Tubos hor.\ y ver.}
\end{equation*}
\underline{Caso}: Otros fluidos, flujo turbulento, otras formas geométricas:
\vspace{-0.3cm}
\begin{equation*}
    \text{Nu} = C\,Ra^a
\end{equation*}
\vspace{-0.3cm}
\underline{Convección transitoria}:
\begin{equation*}
    -q_c = q_s
\end{equation*}
\vspace{-0.4cm}
\begin{equation*}
    -h\,A\,(t_s-t) = m\,C_p\,\frac{dt}{d\theta}
\end{equation*}
\vspace{-0.3cm}
\begin{equation*}
    \theta =
    -\frac{m\,\bar{C_p}}{A\,\bar{h}}\,\ln\left(\frac{t_s-f_f}{t_s-t_i}\right)
\end{equation*}
\underline{Casos particulares}:
Múltiples tubos horizontales y/o verticales:
\vspace{-0.3cm}
\begin{equation*}
    A_T = N_t A_t
\end{equation*}
\vspace{-0.3cm}
\begin{equation*}
    A_t = \pi D_E\,l
\end{equation*}
En placas verticales se debe tomar el volumen por encima de la placa vertical.\\
\underline{Cavidades}: ($L/b>3$)
\vspace{-0.3cm}
\begin{equation*}
    \bar{t} = \frac{T_1 + T_2}{2}
\end{equation*}
\vspace{-0.3cm}
\begin{equation*}
    \text{Gr} = \frac{g\,b^3\,\beta\,(T_1 - T_2)}{\gamma^2}
\end{equation*}
Conducción pura ($\text{Gr}<2000$):
\vspace{-0.3cm}
\begin{equation*}
    \text{Nu} = 1
\end{equation*}
Convección natural en régimen laminar ($\num{2e4} < \text{Gr} < \num{2e5}$):
\vspace{-0.3cm}
\begin{equation*}
    \text{Nu} = 0.18\,\text{Gr}^{\frac{1}{4}}\,
    \left(\frac{L}{b}\right)^{-\frac{1}{9}}
\end{equation*}
Convección natural en régimen turbulento ($\num{2e5} < \text{Gr} < \num{2e7}$):
\vspace{-0.3cm}
\begin{equation*}
    \text{Nu} = 0.065\,\text{Gr}^{\frac{1}{3}}\,
    \left(\frac{L}{b}\right)^{-\frac{1}{9}}
\end{equation*}

\vspace{-0.5cm}
\section*{CONVECCIÓN FORZADA}
\vspace{-0.6cm}
\begin{equation*}
    q = h\,A\,(t_{wi} - \bar{t});\quad
    q_S = \dot{m}\,C_p\,(t_o - t_i)
\end{equation*}
\begin{equation*}
    v\,\rho = \frac{\dot{m}}{A_T} = G;\quad
    \text{Re} = \frac{G\,D_I}{\mu}
\end{equation*}
\vspace{-0.3cm}
\tiny
\begin{itemize}
    \item $\text{Re} \le 2100$, el régimen es \textbf{laminar}.
    \item $\text{Re} > 2100$, el régimen es \textbf{turbulento}.
\end{itemize}
\scriptsize
\underline{Tubo único (flujo interior)}:\\
\underline{Caso}: Régimen turbulento (Ecuaciones de
\emph{Dittus}-\emph{Boelter})
\vspace{-0.3cm}
\begin{equation*}
    \text{Nu}_F = 0.023\,\text{Re}^{0.8}\,\text{Pr}^{0.33}
\end{equation*}
\vspace{-0.3cm}
\begin{equation*}
    t_\text{F} = \frac{t_\text{INT} + \bar{t}_{\infty}}{2}
\end{equation*}
\begin{equation*}
    \text{Nu} = 0.023\,\text{Re}^{0.8}\,\text{Pr}^{0.4}
    \quad\text{(Calentamiento)}
\end{equation*}
\vspace{-0.3cm}
\begin{equation*}
    \text{Nu} = 0.023\,\text{Re}^{0.8}\,\text{Pr}^{0.3}
    \quad\text{(Enfriamiento)}
\end{equation*}
\begin{equation*}
    \bar{t}_\infty = \frac{t_1 + t_2}{2}
\end{equation*}
\vspace{-0.3cm}
\underline{Caso}: Gases
\begin{equation*}
    \text{Pr} = 0.74
\end{equation*}
\begin{equation*}
    \text{Nu} = 0.021\,\text{Re}^{0.8}
\end{equation*}
\underline{Caso}: Flujo isotérmico (vapor)
\vspace{-0.3cm}
\begin{equation*}
    h = 0.023\,
    \left(\frac{G^{0.8}}{D^{0.2}}\right)
    \left(\frac{Cp^{0.4}\,k^{0.6}}{\mu^{0.4}}\right)
\end{equation*}
\underline{Caso}: Fluido muy viscoso ($\text{Re} \le 8000$) (Ecuación de
\emph{Sieder} y \emph{Tate})
\vspace{-0.3cm}
\begin{equation*}
    \text{Nu} = 0.027\,\text{Re}^{0.8}\,\text{Pr}^{0.333}\,
    \left(\frac{\mu}{\mu_S}\right)^{0.14}
\end{equation*}
\vspace{-0.3cm}
\tiny
\begin{itemize}
    \item $\mu$: Viscosidad a la temperatura media del fluido.
    \item $\mu_S$: Viscosidad a la temperatura de superficie.
\end{itemize}
\scriptsize
\underline{Caso}: Régimen laminar
\vspace{-0.3cm}
\begin{equation*}
    \text{Nu} = 2.0\,
    \left(\frac{\dot{m}\,Cp}{k\,L}\right)^{\frac{1}{3}}
    \left(\frac{\mu}{\mu_S}\right)^{0.14}
\end{equation*}
\underline{Caso}: Agua
\vspace{-0.3cm}
\begin{equation*}
    \frac{\mu}{\mu_S} = 1.0
\end{equation*}
\underline{Tubo único (flujo exterior)}:\\
\underline{Caso}: Régimen turbulento\\
Para líquidos:
\vspace{-0.3cm}
\begin{equation*}
    \text{Nu}_\text{F} = \text{Pr}_\text{F}^{0.3}\,
    (0.35+0.47\text{Re}_\text{F}^{0.52})
\end{equation*}
Para gases:
\vspace{-0.3cm}
\begin{equation*}
    \text{Nu}_\text{F} = 0.26\,\text{Pr}_\text{F}^{0.3}\,
    \text{Re}_\text{F}^{0.6}
\end{equation*}
\vspace{-0.3cm}
\begin{equation*}
    t_\text{F} = \frac{t_\text{WE} + \bar{t}_{\infty}}{2}
\end{equation*}
\vspace{-0.2cm}
\begin{equation*}
    \bar{t}_\infty = \frac{t_i + t_o}{2}
\end{equation*}
\underline{Caso}: Aire y gases diatómicos
\vspace{-0.3cm}
\begin{equation*}
    \text{Nu} = 0.32 + 0.43\,\text{Re}^{0.52}
\end{equation*}
\vspace{-0.3cm}
\begin{equation*}
    \text{Nu} = 0.45 + 0.33\,\text{Re}^{0.56}
\end{equation*}
\vspace{-0.2cm}
\begin{equation*}
    \text{Nu} = 0.24\,\text{Re}^{0.6}
\end{equation*}
\underline{Caso}: Cambiadores de calor de doble tubo o tubos concéntricos
(Ecuación de \emph{Davis})
\vspace{-0.3cm}
\begin{equation*}
    \begin{split}
        \frac{h}{Cp\,G} = 0.029\,
        \text{Re}^{-0.2}\,
        \text{Pr}^{-\frac{2}{3}}\\
        \left(\frac{\mu}{\mu_s}\right)^{0.14}
        \left(\frac{D_\text{E}}{D_\text{I}}\right)^{0.25}
    \end{split}
\end{equation*}
\underline{Caso}: Régimen laminar\\
Para líquidos ($0.1 < \text{Re} < 200$):
\vspace{-0.3cm}
\begin{equation*}
    \text{Nu}_\text{F} = 0.86\,\text{Pr}_\text{F}^{0.3}\,
    \text{Re}_\text{F}^{0.43}
\end{equation*}
Para líquidos ($\text{Re} > 200$) y gases ($0.1 < \text{Re} < 1000$):
\vspace{-0.3cm}
\begin{equation*}
    \text{Nu}_\text{F} = \text{Pr}_\text{F}^{0.3}\,
    (0.35 + 0.47\,\text{Re}_\text{F}^{0.52})
\end{equation*}
\underline{Caso}: Aire
\vspace{-0.3cm}
\begin{equation*}
    \text{Nu}_\text{F} = 0.24\,\text{Re}^{0.6}
\end{equation*}
\underline{Haces de tubos (flujo exterior)}\\
\underline{(Método de \emph{Crimson})}:
\vspace{-0.3cm}
\begin{equation*}
    \text{Nu} = C\,\text{Re}_\text{max}^n
\end{equation*}
\vspace{-0.3cm}
\begin{equation*}
    \text{Re}_\text{max} = \frac{v_\text{max}\,D_E}{\nu}
\end{equation*}
\underline{Arreglo en linea}:
\vspace{-0.3cm}
\begin{equation*}
    P_\text{min} = a - D
\end{equation*}
\vspace{-0.3cm}
\begin{equation*}
    v_\text{max} = \frac{V_\infty\,a}{P_\text{min}}
\end{equation*}
\underline{Arreglo escalonado}:
\vspace{-0.3cm}
\begin{equation*}
    P_\text{min1} = \frac{a - D}{2}
\end{equation*}
\vspace{-0.3cm}
\begin{equation*}
    P_\text{min2} = \sqrt{\left(\frac{a}{2}\right)^2 + b^2} - D
\end{equation*}
\vspace{-0.3cm}
\begin{equation*}
    v_\text{max} = \frac{V_\infty\,(a/2)}{\min(P_\text{min1}, P_\text{min2})}
\end{equation*}
Para 10 o mas tubos:
\vspace{-0.3cm}
\begin{equation*}
    h_o = \frac{\text{Nu}\,k}{D}
\end{equation*}

\vspace{-0.5cm}
\section*{CAMBIADORES DE CALOR}
\vspace{-0.3cm}
\begin{equation*}
    q_s = \dot{m}\,C_p\,(t_o - t_i)
\end{equation*}
\vspace{-0.2cm}
\begin{equation*}
    q_l = \dot{m}\,(\Delta H)
\end{equation*}
\vspace{-0.2cm}
\begin{equation*}
    q = U\,A\,\Delta t
\end{equation*}
\underline{Coeficiente global}:\\
Pared vertical:
\vspace{-0.3cm}
\begin{equation*}
    U = \dfrac{1}{\frac{1}{h_i} + \frac{\Delta x_p}{k_p} + \frac{1}{h_o}}
\end{equation*}
Conductor cilíndrico:
\vspace{-0.3cm}
\begin{equation*}
    U_o = \dfrac{1}{\frac{1}{h_o} + \frac{D_e}{2k_p}\ln\left(\frac{D_e}{D_i}\right) + \frac{1}{h_i}\left(\frac{D_e}{D_i}\right)}
\end{equation*}
\vspace{-0.3cm}
\begin{equation*}
    U_i = \dfrac{1}{\frac{1}{h_i} + \frac{D_i}{2k_p}\ln\left(\frac{D_e}{D_i}\right) + \frac{1}{h_o}\left(\frac{D_i}{D_e}\right)}
\end{equation*}
\vspace{-0.3cm}
\begin{equation*}
    \frac{U_o}{U_i} = \frac{A_i}{A_o} = \frac{D_i}{D_e}
\end{equation*}
\underline{Incrustaciones}
\vspace{-0.3cm}
\begin{equation*}
    \frac{1}{U_{d}} = \frac{1}{U} + \sum R_d
\end{equation*}
\underline{Gradiente de temperatura}
\vspace{-0.3cm}
\begin{equation*}
    \Delta t_{log} = \frac{\Delta t_1 - \Delta t_2}{\ln\left(\frac{\Delta t_1}{\Delta t_2}\right)}
\end{equation*}
\underline{Factor de corrección}
\vspace{-0.3cm}
\begin{equation*}
    \Delta t = F_c\,\Delta t_{log}
\end{equation*}
Para cambiadores 1:1 y condensadores el factor de corrección $F_c = 1$.\\
\underline{Eficacia de un cambiador}:\\
$Z > 1$:
\vspace{-0.2cm}
\begin{equation*}
    \eta = \frac{t_{c2} - t_{c1}}{t_{h2} - t_{c1}}
\end{equation*}
$Z < 1$:
\vspace{-0.3cm}
\begin{equation*}
    \eta = \frac{t_{h1} - t_{h2}}{t_{h1} - t_{c1}}
\end{equation*}
Numero de tubos:
\vspace{-0.3cm}
\begin{equation*}
    N_T = \frac{L_T}{L_1}
        = \frac{\dot{m}_T}{\dot{m}_1}
        = \frac{A_T}{A_1}
\end{equation*}
Caso 1:1:
\vspace{-0.3cm}
\begin{equation*}
    N_T\Biggr|_q = N_T\Biggr|_{\dot{m}}
\end{equation*}
Caso 1:2:
\vspace{-0.3cm}
\begin{equation*}
    N_T\Biggr|_q = 2\,N_T\Biggr|_{\dot{m}}
\end{equation*}
Caso 1:3:
\vspace{-0.3cm}
\begin{equation*}
    N_T\Biggr|_q = 3\,N_T\Biggr|_{\dot{m}}
\end{equation*}
\underline{Calculo de $h$ de condensación}:\\
Para superficies verticales:\\
Para placas:
\vspace{-0.3cm}
\begin{equation*}
    h = 1.13\,\left(\frac{k_f^3\,\rho_f^2\,g\,\Delta H}{L\,\mu_f\,(t_v - t_s)}\right)^{\frac{1}{4}}
\end{equation*}
Para tubos:
\vspace{-0.3cm}
\begin{equation*}
    h = 1.18\,\left(\frac{k_f^3\,\rho_f^2\,g\,\pi\,D_E}{\mu_f\,W}\right)^{\frac{1}{3}}
\end{equation*}
Para tubos horizontales:
\vspace{-0.3cm}
\begin{equation*}
    h = 0.725\,\left(\frac{k_f^3\,\rho_f^2\,g\,\Delta H}{N^{2/3}\,D_E\,\mu_f\,(t_v - t_s)}\right)^{\frac{1}{4}}
\end{equation*}
Las propiedades del fluido se calculan a la temperatura media de la película
condensada:
\vspace{-0.2cm}
\begin{equation*}
    t_f = t_v - \frac{3}{4}\,(t_v - t_s)
\end{equation*}
\underline{Método NUT}:
\vspace{-0.3cm}
\begin{equation*}
    \epsilon = \frac{q_r}{q_{\text{max}}}
\end{equation*}
\vspace{-0.3cm}
\begin{equation*}
    q_{\text{max}} = C_{\text{min}}\,(t_{h1} - t_{c1})
\end{equation*}
\vspace{-0.2cm}
\begin{equation*}
    C_{\text{min}} = \min(C_h, C_c)
\end{equation*}
\vspace{-0.2cm}
\begin{equation*}
    C_h = \dot{m}_h\,C_{ph}
\end{equation*}
\vspace{-0.2cm}
\begin{equation*}
    C_c = \dot{m}_c\,C_{pc}
\end{equation*}
\vspace{-0.2cm}
\begin{equation*}
    \text{NUT} = \frac{U\,A}{C_{\text{min}}}
\end{equation*}
\vspace{-0.2cm}
\begin{equation*}
    C = \frac{C_{\text{min}}}{C_{\text{max}}}
\end{equation*}
\underline{Superficies extendidas}:
\vspace{-0.3cm}
\begin{equation*}
    q = N_A\,h_o\,(t_o - t_{\infty})(A_{s/a1} + \eta\,A_{a1})
\end{equation*}
\vspace{-0.3cm}
\tiny
\begin{itemize}
    \item $N_A$: Numero de aletas.
    \item $A_{s/a1}$: Área libre entre dos aletas.
    \item $A_{a1}$: Área de una aleta.
\end{itemize}

\scriptsize
\vspace{-0.5cm}
\section*{RADIACIÓN}
\vspace{-0.5cm}
\begin{equation*}
    \begin{split}
    \sigma
        &= \num{0.173e-8}\,\left[\frac{btu}{h\,pie^2\,^\circ R^4}\right]\\
        &= \num{4.92e-8}\,\left[\frac{kcal}{h\,m^2\,K^4}\right]
    \end{split}
\end{equation*}
\vspace{-0.3cm}
\begin{equation*}
    q_{1,2} = A_1\,\phi_{1,2}\,\sigma\,(T_1^4 - T_2^4)
\end{equation*}
\vspace{-0.1cm}
\begin{equation*}
    q_{\text{solar}} = A\,I_{\text{solar}}\,\alpha_{\text{abs}}
\end{equation*}
\vspace{-0.1cm}
\begin{equation*}
    \phi_{1,2} = \dfrac{1}{\dfrac{1}{F_{1,2}}+\dfrac{1}{\epsilon_1}-1
    +\dfrac{A_1}{A_2}\left(\dfrac{1}{\epsilon_2}-1\right)}
\end{equation*}
\underline{Caso}: Superficies grises paralelas de igual área.
\vspace{0.1cm}
\begin{equation*}
    \phi_{1,2} = \dfrac{1}{\dfrac{1}{\epsilon_1}+\dfrac{1}{\epsilon_2}-1}
\end{equation*}
\underline{Caso}: Superficie pequeña rodeada totalmente por otra mas grande
\vspace{-0.3cm}
\begin{equation*}
    \phi_{1,2} = \epsilon_1
\end{equation*}
\underline{Propiedades del factor de forma}:\\
Subdivisión de la superficie emisora:
\vspace{-0.3cm}
\begin{equation*}
    A_1 = \sum_{i=1}^n A_i
\end{equation*}
Subdivisión de la superficie receptora:
\vspace{-0.3cm}
\begin{equation*}
    A_2 = \sum_{i=1}^n A_i
\end{equation*}
Espacios cerrados:
\vspace{-0.3cm}
\begin{equation*}
    F_{1,1} + F_{1,2} + F_{1,3} + \cdots + F_{1,n} = 1
\end{equation*}
Teorema de la reciprocidad:
\vspace{-0.3cm}
\begin{equation*}
    A_1\,F_{1,2} = A_2\,F_{2,1}
\end{equation*}
\end{multicols}

\end{document}

