\chapter{GENERALIDADES}

\section{Conceptos de termodinámica}

\subsection{Calor}
Es la energía transferida a un sistema o sus alrededores debido a la diferencia
de temperatura.
\subsection{Temperatura}
En un parámetro de estado que mide el estado de movilidad de los átomos y
moléculas que componen el sistema.
\subsection{Energía interna}
Es una forma de energía cinética de los átomos y moléculas.
\subsection{Parámetros de estado}
Son las magnitudes que se emplean para describir el estado de un sistema
termodinámico.
\subsection{Proceso}
Es cualquier cambio en el estado de un sistema.
\subsection{Entalpía}
Cantidad de energía que un sistema termodinámico intercambia con su medio
ambiente en condiciones de presión constante.
\subsection{Primera ley de la termodinámica}
Es un enunciado de conservación de energía, que nos dice que un sistema puede
intercambiar energía con su entorno mediante la transmisión de calor y la
realización de trabajo.
\subsection{Segunda ley de la termodinámica}
Es un hecho experimental que nos dice que no es posible ningún proceso
espontáneo cuyo único resultado sea el paso de calor de un recinto a otro de
mayor temperatura.
\subsection{Entropía}
Es la magnitud física que mide la parte de la energía que no puede utilizarse
para producir trabajo.

\section{Mecanismos de transferencia de calor}

\subsection{Conducción}
Es un mecanismo de transferencia de calor propio de los medios solidos a nivel
atómico.

\subsubsection{Ley de \emph{Fourier}}
\begin{equation}
    q=k\,A\,\frac{\Delta T}{\Delta x}
\label{conduccion}
\end{equation}

Donde:
\begin{itemize}
    \item \emph{$k$}: Coeficiente de conducción.
    \item \emph{$A$}: Área de transferencia de calor.
    \item \emph{$\Delta T$}: Gradiente de temperatura ($t_1-t_0$).
    \item \emph{$\Delta x$}: Espesor del campo.
\end{itemize}

\subsection{Convección}
Es un mecanismo de transferencia de calor propio de los medios fluidos.
\begin{equation}
    q=h\,A\,\Delta T
\label{conveccion}
\end{equation}

Donde:
\begin{itemize}
    \item \emph{$h$}: Coeficiente de convección.
    \item \emph{$A$}: Área de transferencia de calor.
    \item \emph{$\Delta T$}: Gradiente de temperatura ($t_1-t_0$).
\end{itemize}

\subsection{Radiación}

\subsubsection{Ley de \emph{Stefan}-\emph{Boltzmann}}
\begin{equation}
    q=A\,\sigma\,T^4
\label{radiacion}
\end{equation}

Donde:
\begin{itemize}
    \item \emph{$A$}: Área de transferencia de calor.
    \item \emph{$\sigma$}: Constante de \emph{Boltzmann}.
    \item \emph{$T$}: Temperatura del cuerpo radiante.
\end{itemize}

\section{Analogías}
\begin{equation}
    \text{Flujo}=\frac{\text{Potencial}}{\text{Resistencia}}
\label{analogias}
\end{equation}

\begin{table}[!h]
\begin{center}
\begin{tabular}{|>{\centering}m{2.0cm}<{\centering}
                |>{\centering}m{2.0cm}<{\centering}
                |>{\centering}m{3.6cm}<{\centering}
                |>{\centering}m{3.6cm}<{\centering}|}
\hline
\textbf{Área} & \textbf{Flujo} & \textbf{Potencial} & \textbf{Resistencia}
\tabularnewline \hline
Térmica & Calor & Diferencia de temperatura & Resistencia al calor
\tabularnewline \hline
Eléctrica & Corriente  & Diferencia de voltaje & Resistencia a la corriente
\tabularnewline \hline
Fluida & Caudal & Diferencia de presiones & Resistencia al caudal
\tabularnewline \hline
Trabajo mecánico & Trabajo & Diferencia de presiones & Resistencia al trabajo
\tabularnewline \hline
Económica & Ganancia & Mercado & Resistencia a la ganancia
\tabularnewline \hline
\end{tabular}
\caption{Ejemplos de analogías en diversas áreas.}
\label{cuadro1}
\end{center}
\end{table}

\begin{equation*}
\def\arraystretch{1.4}
\begin{array}{@{}cll@{}}
\toprule
\text{Problema de calor} & \rightleftarrows & \text{Problema eléctrico} \\
\cmidrule(l){1-3}
q=\dfrac{\Delta T}{R_T} & & I=\dfrac{\Delta V}{R} \\
\bottomrule
\end{array}
\end{equation*}

