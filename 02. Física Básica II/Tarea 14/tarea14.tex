\documentclass[letter,11pt]{article}

\usepackage[spanish,es-nodecimaldot]{babel}
\usepackage[utf8]{inputenc}

\usepackage{lmodern}
\usepackage[T1]{fontenc}
\usepackage{textcomp}

\usepackage{framed}
\usepackage[svgnames]{xcolor}
\colorlet{shadecolor}{Gainsboro!50}

\usepackage[shortlabels]{enumitem}
\usepackage{graphicx}
\usepackage{pstricks}

\usepackage{anysize}
\marginsize{3cm}{2cm}{2cm}{3cm}

\usepackage{siunitx}
\usepackage{amsmath}
\usepackage{array}
\usepackage{alltt}

\usepackage{fancyhdr}
\usepackage{lastpage}
\pagestyle{fancy}
\fancyhf{}
\fancyhead[LE,RO]{Física Básica II}
\fancyfoot[CO,CE]{\thepage\ de \pageref{LastPage}}

\special{papersize=215.9mm,279.4mm}

\usepackage[
    pdfauthor={Carlos Eduardo Caballero Burgoa},%
    pdftitle={Física Básica II},%
    pdfsubject={Tarea 14},%
    colorlinks,%
    citecolor=black,%
    filecolor=black,%
    linkcolor=black,%
    urlcolor=black,
    breaklinks]{hyperref}
\usepackage{breakurl}

\newcommand{\blankpage}{
\newpage
\thispagestyle{empty}
\mbox{}
\newpage
}

\renewcommand{\arraystretch}{1.2}

\begin{document}

\begin{center}
    {\Large \bf{Tarea \#14}}
\end{center}

\section{Ejercicio 1}

Demostrar que una solución de la ecuación:

\begin{equation}
    x'' + \omega^2 x = 0
\end{equation}

es:

\begin{equation}
    x = A \cdot cos(\omega t)
\end{equation}

Si las condiciones iniciales son: $x = A$, $x' = 0$ para $t = 0$.

\vspace{0.5cm}
\textbf{Solución:} \\

\begin{equation*}
    x'' + \omega^2 x = 0
\end{equation*}
\begin{equation*}
    x'' = - \omega^2 x
\end{equation*}
\begin{equation*}
    x' \frac{dx'}{dx} = - \omega^2 x
\end{equation*}
\begin{equation*}
    x' dx' = - \omega^2 x dx
\end{equation*}

Considerando las condiciones iniciales: $x = A$, $x' = 0$ en $t = 0$:

\begin{equation*}
    \int_{0}^{x'} x' dx' = - \omega^2 \int_{A}^{x} x dx
\end{equation*}
\begin{equation*}
    \frac{x'^2}{2} \Biggr|_{0}^{x'} = - \omega^2 \frac{x^2}{2} \Biggr|_{A}^{x}
\end{equation*}
\begin{equation*}
    \frac{x'^2}{2} = - \omega^2 \left( \frac{x^2}{2} - \frac{A^2}{2} \right)
\end{equation*}
\begin{equation*}
    x'^2 = - \omega^2 (A^2 - x^2)
\end{equation*}
\begin{equation*}
    x' = \omega \sqrt{A^2 - x^2}
\end{equation*}
\begin{equation*}
    \frac{dx}{dt} = \omega \sqrt{A^2 - x^2}
\end{equation*}
\begin{equation*}
    \omega dt = \frac{dx}{\sqrt{A^2 - x^2}}
\end{equation*}
\begin{equation*}
    \omega \int_{0}^{t} dt = \int_{A}^{x} \frac{dx}{\sqrt{A^2 - x^2}}
\end{equation*}

Realizando el siguiente cambio de variable:

\begin{equation*}
    x = A \cdot cos(\theta)
\end{equation*}
\begin{equation*}
    dx = - A \cdot sen(\theta) d\theta
\end{equation*}

Obtenemos:

\begin{equation*}
    \omega t = \int \frac{-A sen(\theta) d\theta}{\sqrt{A^2 - (A \cdot cos(\theta))^2}}
\end{equation*}
\begin{equation*}
    \omega t = \int \frac{- sen(\theta) d\theta}{\sqrt{1 - cos^2(\theta)}}
\end{equation*}
\begin{equation*}
    \omega t = \int - d\theta
\end{equation*}
\begin{equation*}
    \omega t = - \theta
\end{equation*}

Sabiendo que $cos(\theta) = cos(-\theta)$:

\begin{equation*}
    \omega t = arccos\left(\frac{x}{A}\right) \Biggr|_{A}^{x}
\end{equation*}
\begin{equation*}
    \omega t = arccos\left(\frac{x}{A}\right) - arccos\left(\frac{A}{A}\right)
\end{equation*}
\begin{equation*}
    \omega t = arccos\left(\frac{x}{A}\right)
\end{equation*}
\begin{equation*}
    cos(\omega t) = \frac{x}{A}
\end{equation*}
\begin{equation}
    x = A \cdot cos(\omega t)
\end{equation}

\section{Ejercicio 2}

Encontrar una solución a la ecuación:

\begin{equation}
    x'' + \omega^2 x = 0
\end{equation}

es:

\begin{equation}
    x = A \cdot cos(\omega t)
\end{equation}

Con condiciones iniciales: $x = 0$, $x' = x'_0$ para $t = 0$.

\vspace{0.5cm}
\textbf{Solución:} \\

\begin{equation*}
    x'' + \omega^2 x = 0
\end{equation*}
\begin{equation*}
    x'' = - \omega^2 x
\end{equation*}
\begin{equation*}
    x' \frac{dx'}{dx} = - \omega^2 x
\end{equation*}
\begin{equation*}
    x' dx' = - \omega^2 x dx
\end{equation*}

Considerando las condiciones iniciales: $x = 0$, $x' = x'_0$ en $t = 0$:

\begin{equation*}
    \int_{x'_0}^{x'} x' dx' = - \omega^2 \int_{0}^{x} x dx
\end{equation*}
\begin{equation*}
    \frac{x'^2}{2} \Biggr|_{x'_0}^{x'} = - \omega^2 \frac{x^2}{2} \Biggr|_{0}^{x}
\end{equation*}
\begin{equation*}
    \frac{x'^2}{2} - \frac{x'^2_0}{2} = - \omega^2 \frac{x^2}{2}
\end{equation*}
\begin{equation*}
    x'^2 - x'^2_0 = - \omega^2 x^2
\end{equation*}
\begin{equation*}
    x' = \sqrt{x'^2_0 - \omega^2 x^2}
\end{equation*}
\begin{equation*}
    \frac{dx}{dt} = \sqrt{x'^2_0 - \omega^2 x^2}
\end{equation*}
\begin{equation*}
    dt = \frac{dx}{\sqrt{x'^2_0 - \omega^2 x^2}}
\end{equation*}
\begin{equation*}
    \int_{0}^{t} dt = \int_{0}^{x} \frac{dx}{\sqrt{x'^2_0 - \omega^2 x^2}}
\end{equation*}
\begin{equation*}
    t = \int_{0}^{x} \frac{dx}{\sqrt{x'^2_0 \left(1 - \frac{\omega^2 x^2}{x'^2_0}\right)}}
\end{equation*}
\begin{equation*}
    t = \frac{1}{x'_0} \int_{0}^{x} \frac{dx}{\sqrt{1 - \frac{\omega^2 x^2}{x'^2_0}}}
\end{equation*}

Realizando el siguiente cambio de variable:

\begin{equation*}
    \frac{\omega x}{x'_0} = sen(\theta)
\end{equation*}
\begin{equation*}
    \frac{\omega dx}{x'_0} = cos(\theta) d\theta
\end{equation*}

Obtenemos:

\begin{equation*}
    t = \frac{1}{x'_0} \int \frac{x'_0 cos(\theta) d\theta}{\omega \sqrt{1 - sen^2(\theta)}}
\end{equation*}
\begin{equation*}
    t = \frac{1}{\omega} \int \frac{cos(\theta) d\theta}{\sqrt{1 - sen^2(\theta)}}
\end{equation*}
\begin{equation*}
    t = \frac{1}{\omega} \int d\theta
\end{equation*}
\begin{equation*}
    t = \frac{1}{\omega} \theta
\end{equation*}
\begin{equation*}
    t = \frac{1}{\omega} \cdot arcsen\left(\frac{\omega x}{x'_0} \right) \Biggr|_{0}^{x}
\end{equation*}
\begin{equation*}
    t = \frac{1}{\omega} \left( arcsen\left(\frac{\omega x}{x'_0}\right) - arcsen\left(\frac{\omega \cdot 0}{x'_0}\right) \right)
\end{equation*}
\begin{equation*}
    \omega t = arcsen\left(\frac{\omega x}{x'_0}\right)
\end{equation*}
\begin{equation*}
    sen(\omega t) = \frac{\omega x}{x'_0}
\end{equation*}
\begin{equation}
    x = \frac{x'_0}{\omega} \cdot sen(\omega t)
\end{equation}

\end{document}

