\documentclass[letter,11pt]{article}

\usepackage[spanish,es-nodecimaldot]{babel}
\usepackage[utf8]{inputenc}

\usepackage{lmodern}
\usepackage[T1]{fontenc}
\usepackage{textcomp}

\usepackage{framed}
\usepackage[svgnames]{xcolor}
\colorlet{shadecolor}{Gainsboro!50}

\usepackage[shortlabels]{enumitem}
\usepackage{graphicx}
\usepackage{pstricks}

\usepackage{anysize}
\marginsize{3cm}{2cm}{2cm}{3cm}

\usepackage{siunitx}
\usepackage{amsmath}
\usepackage{array}
\usepackage{alltt}

\usepackage{fancyhdr}
\usepackage{lastpage}
\pagestyle{fancy}
\fancyhf{}
\fancyhead[LE,RO]{Física Básica II}
\fancyfoot[CO,CE]{\thepage\ de \pageref{LastPage}}

\special{papersize=215.9mm,279.4mm}

\usepackage[
    pdfauthor={Carlos Eduardo Caballero Burgoa},%
    pdftitle={Física Básica II},%
    pdfsubject={Tarea 16},%
    colorlinks,%
    citecolor=black,%
    filecolor=black,%
    linkcolor=black,%
    urlcolor=black,
    breaklinks]{hyperref}
\usepackage{breakurl}

\newcommand{\blankpage}{
\newpage
\thispagestyle{empty}
\mbox{}
\newpage
}

\renewcommand{\arraystretch}{1.2}

\begin{document}

\begin{center}
    {\Large \bf{Tarea \#16}}
\end{center}

En un sistema masa-resorte donde: $k = 80[N/m]$ y $m = 0.2 [kg]$, la ecuación
diferencial es:

\begin{equation}
    x'' + \frac{k}{m} x = 0
\end{equation}

Si las condiciones iniciales son $v_0 = 0$ y $x_0 = 0.2 [m]$, cuando
$t = 0$.

\begin{enumerate}[a)]
    \item Calcular la posición, velocidad, aceleración, energía cinética,
        potencial y la total en función del tiempo.
    \item Calcular la energía cinética, energía potencial y la energía total
        para $x = -0.1 [m]$.
\end{enumerate}

\vspace{0.5cm}
\textbf{Solución:} \\

\textbf{(a)} \\

Calculando $\omega$:

\begin{equation}
    \omega = \sqrt{\frac{k}{m}} = \sqrt{\frac{80}{0.2}} = \sqrt{400} = 20 [s^{-1}]
\end{equation}

Hallamos $A$ y $\phi$, a partir de la ecuación de posición y velocidad:

\begin{equation*}
    x = A \cdot cos(\omega t - \phi)
\end{equation*}
\begin{equation*}
    0.2 = A \cdot cos(-\phi) = A \cdot cos(\phi)
\end{equation*}
\begin{equation*}
    v = -A \omega \cdot sen(\omega t - \phi)
\end{equation*}
\begin{equation*}
    0 = -A (20) \cdot sen(-\phi) = A \cdot sen(\phi)
\end{equation*}

\begin{equation*}
    sen(\phi) = 0
\end{equation*}
\begin{equation*}
    \phi = arcsen(0) = 0
\end{equation*}
\begin{equation*}
    A = \frac{0.2}{cos(\phi)} = \frac{0.2}{cos(0)} = 0.2 [m]
\end{equation*}

Por tanto:

\begin{equation}
    x = A \cdot cos (\omega t - \phi) = 0.2 \cdot cos (20t) [m]
\end{equation}
\begin{equation}
    v = -A \omega \cdot sen (\omega t - \phi) = -4 \cdot sen (20t) [m/s]
\end{equation}
\begin{equation}
    a = -A \omega^2 \cdot cos (\omega t - \phi) = -80 \cdot cos(20t) [m/s^2]
\end{equation}
\begin{equation}
    T = \frac{1}{2} m v^2 = \frac{1}{2} (0.2) (-4^2 \cdot sen^2(20t)) = 1.6 \cdot sen^2(20t) [J]
\end{equation}
\begin{equation}
    U = \frac{1}{2} k x^2 = \frac{1}{2} (80) (0.2^2 \cdot cos^2(20t)) = 1.6 \cdot cos^2(20t) [J]
\end{equation}
\begin{equation}
    E = T + U = 1.6 \cdot sen^2(20t) + 1.6 \cdot cos^2(20t) = 1.6 [J]
\end{equation}

\textbf{(b)} \\

Para:

\begin{equation*}
    x = -0.1 [m]
\end{equation*}
\begin{equation*}
    -0.1 = 0.2 \cdot cos(20t)
\end{equation*}
\begin{equation*}
    t = \frac{arccos(-0.5)}{20} = \frac{1}{20} \frac{2 \pi}{3} = \frac{\pi}{30} [s]
\end{equation*}

Calculando la energía cinética:

\begin{equation}
    T = 1.6 \cdot sen^2(20t) = 1.6 \cdot sen^2\left(\frac{20 \pi}{30}\right) = 1.6 \left(\frac{\sqrt{3}}{2}\right)^2 = 1.2 [J]
\end{equation}

Calculando la energía potencial:

\begin{equation}
    U = 1.6 \cdot cos^2(20t) = 1.6 \cdot cos^2\left(\frac{20 \pi}{30}\right) = 1.6 \left(-\frac{1}{2}\right)^2 = 0.4 [J]
\end{equation}

Comprobándose que la energía total se mantuvo invariable:

\begin{equation}
    E = T + U = 1.2 + 0.4 = 1.6 [J]
\end{equation}

\end{document}

