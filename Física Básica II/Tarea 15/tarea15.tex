\documentclass[letter,11pt]{article}

\usepackage[spanish,es-nodecimaldot]{babel}
\usepackage[utf8]{inputenc}

\usepackage{lmodern}
\usepackage[T1]{fontenc}
\usepackage{textcomp}

\usepackage{framed}
\usepackage[svgnames]{xcolor}
\colorlet{shadecolor}{Gainsboro!50}

\usepackage[shortlabels]{enumitem}
\usepackage{graphicx}
\usepackage{pstricks}

\usepackage{anysize}
\marginsize{3cm}{2cm}{2cm}{3cm}

\usepackage{siunitx}
\usepackage{amsmath}
\usepackage{array}
\usepackage{alltt}

\usepackage{fancyhdr}
\usepackage{lastpage}
\pagestyle{fancy}
\fancyhf{}
\fancyhead[LE,RO]{Física Básica II}
\fancyfoot[CO,CE]{\thepage\ de \pageref{LastPage}}

\special{papersize=215.9mm,279.4mm}

\usepackage[
    pdfauthor={Carlos Eduardo Caballero Burgoa},%
    pdftitle={Física Básica II},%
    pdfsubject={Tarea 15},%
    colorlinks,%
    citecolor=black,%
    filecolor=black,%
    linkcolor=black,%
    urlcolor=black,
    breaklinks]{hyperref}
\usepackage{breakurl}

\newcommand{\blankpage}{
\newpage
\thispagestyle{empty}
\mbox{}
\newpage
}

\renewcommand{\arraystretch}{1.2}

\begin{document}

\begin{center}
    {\Large \bf{Tarea \#15}}
\end{center}

Si $x = A \cdot cos(\omega t - \pi / 2)$ representa la ecuación de posición de
un oscilador armónico simple.

\begin{enumerate}[a)]
\item Calcular la velocidad, aceleración, energía cinética, energía potencial y
la energía total en función del tiempo.
\item Calcular la velocidad, aceleración, energía cinética, energía potencial y
la energía total para $x = \pm \frac{A}{\sqrt{2}}$.
\end{enumerate}

\vspace{0.5cm}
\textbf{Solución:} \\

\textbf{(a)}

\begin{equation}
    x = A \cdot cos(\omega t - \pi / 2)
\end{equation}

Derivando la posición, obtenemos la velocidad:

\begin{equation*}
    \frac{dx}{dt} = A \cdot ( -sen(\omega t - \pi / 2)) (\omega)
\end{equation*}
\begin{equation}
    v = -A \omega \cdot sen(\omega t - \pi / 2)
\end{equation}

Derivando la velocidad, obtenemos la aceleración:

\begin{equation*}
    \frac{dv}{dt} = -A \omega \cdot ( cos(\omega t - \pi / 2)) (\omega)
\end{equation*}
\begin{equation}
    a = -A \omega^2 \cdot cos(\omega t - \pi / 2)
\end{equation}

Para la energía cinética:

\begin{equation*}
    T = \frac{1}{2} m v^2 = \frac{1}{2} m ( -A \omega \cdot sen(\omega t - \pi / 2))^2
\end{equation*}
\begin{equation}
    T = \frac{1}{2} m A^2 \omega^2 sen^2(\omega t - \pi/2)
\end{equation}

Para la energía potencial:

\begin{equation*}
    U = \frac{1}{2} k x^2 = \frac{1}{2} k ( A \cdot cos(\omega t - \pi / 2))^2 = \frac{1}{2} k A^2 cos^2(\omega t - \pi / 2)
\end{equation*}
\begin{equation}
    U = \frac{1}{2} m \omega^2 A^2 cos^2(\omega t - \pi / 2)
\end{equation}

Para la energía total:

\begin{equation*}
    T + U = \frac{1}{2} m A^2 \omega^2 sen^2(\omega t - \pi/2) + \frac{1}{2} m \omega^2 A^2 cos^2(\omega t - \pi / 2)
\end{equation*}
\begin{equation*}
    E = \frac{1}{2} m \omega^2 A^2 (sen^2(\omega t - \pi/2) + cos^2(\omega t - \pi/2))
\end{equation*}
\begin{equation}
    E = \frac{1}{2} m \omega^2 A^2
\end{equation}

\textbf{(b)} \\

Siendo: 

\begin{equation}
    x = \pm \frac{A}{\sqrt{2}}
\end{equation}

entonces:

\begin{equation*}
    A \cdot cos(\omega t - \pi/2) = \pm \frac{A}{\sqrt{2}}
\end{equation*}
\begin{equation*}
    cos(\omega t - \pi/2) = \pm \frac{1}{\sqrt{2}}
\end{equation*}
\begin{equation*}
    \omega t - \pi/2 = arccos \left( \pm \frac{1}{\sqrt{2}} \right)
\end{equation*}
\begin{equation*}
    \omega t - \frac{\pi}{2} = arccos \left( \pm \frac{1}{\sqrt{2}} \right)
\end{equation*}

Para la velocidad:

\begin{equation*}
    v = -A \omega \cdot sen(\omega t - \pi/2) = -A \omega \cdot sen\left(arccos\left(\pm \frac{1}{\sqrt{2}}\right)\right)
\end{equation*}
\begin{equation*}
    v = -A \omega \sqrt{1 - \left(\pm \frac{1}{\sqrt{2}}\right)^2} = -A \omega \sqrt{1 - \frac{1}{2}}
\end{equation*}
\begin{equation}
    v = -A \omega \sqrt{\frac{1}{2}}
\end{equation}

Para la aceleración:

\begin{equation*}
    a = -A \omega^2 \cdot cos(\omega t - \pi / 2) = -A \omega^2 \cdot cos\left(arccos\left(\pm \frac{1}{\sqrt{2}}\right)\right)
\end{equation*}
\begin{equation}
    a = \mp \frac{A \omega^2}{\sqrt{2}}
\end{equation}

Para la energía cinética:

\begin{equation*}
    T = \frac{1}{2} m A^2 \omega^2 sen^2(\omega t - \pi/2) = \frac{1}{2} m A^2 \omega^2 sen^2\left(arccos\left(\pm \frac{1}{\sqrt{2}}\right)\right)
\end{equation*}
\begin{equation*}
    T = \frac{1}{2} m A^2 \omega^2 \sqrt{1 - \left(\pm \frac{1}{\sqrt{2}}\right)^2} = \frac{1}{2} m A^2 \omega^2 \left(1 - \frac{1}{2}\right)
\end{equation*}
\begin{equation}
    T = \frac{1}{4} m A^2 \omega^2
\end{equation}

Para la energía potencial:

\begin{equation*}
    U = \frac{1}{2} m \omega^2 A^2 cos^2(\omega t - \pi / 2) = \frac{1}{2} m \omega^2 A^2 cos^2\left(arccos\left(\pm \frac{1}{\sqrt{2}}\right)\right)
\end{equation*}
\begin{equation*}
    U = \frac{1}{2} m \omega^2 A^2 \left(\pm \frac{1}{\sqrt{2}}\right)^2 = \frac{1}{2} m \omega^2 A^2 \left(\frac{1}{2}\right)
\end{equation*}
\begin{equation}
    U = \frac{1}{4} m \omega^2 A^2
\end{equation}

Para la energía total:

\begin{equation*}
    T + U = \frac{1}{4} m A^2 \omega^2 + \frac{1}{4} m \omega^2 A^2
\end{equation*}
\begin{equation}
    E = \frac{1}{2} m \omega^2 A^2
\end{equation}

\end{document}

